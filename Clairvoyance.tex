Using the hiding property of a hash function, we might argue that the government might also guarantee for the immutability of their secret data that they do not want to publish. More precisely, assume that the government has a list $d_1,d_2,\ldots ,d_k$ of documents they wish to keep secret, but want to assure the general public that the documents themselves will not be changed not manipulated. For this, the government could simply create the Merkle tree $MT(d_1,\ldots ,d_k)$ that contains the hash values $h(d_1),\ldots ,h(d_k)$ as its leaves, and publish this tree as the safety certificate. If at a later date there is a dispute regarding some issue pertaining to document $d_i$ (e.g. in case that $d_i$ is a contract under question), a judicial body might require the government agency to make this particular document public, and show its membership in the Merkle tree $MT(d_1,\ldots ,d_k)$. In case that the document was manipulated in the mean time, this can now be detected, as the leave hash will not match the path to the root of the tree. Note that this still allows the documents different from $d_i$ to remain secret.

One problem with this solution is that since the documents $d_1,\ldots ,d_k$ are not made public, a malicious government agent might publish two versions of the document: one that suits him and does not reflect signs of foul play, and another that holds the actual true. For example, suppose that the government keeps secret the certificates stating whether the taxes paid by some entity are in order or not. Assume also that a certificate for an entity X was issued stating that the taxes are not in order. A malicious agent is then bribed to also issue a certificate that the taxes are in order, together with the proof of the inclusion in the Merkle tree (note that this can even be the same tree containing the certificates showing the deficiency of the taxes of X). If X is at a later date asked to prove the status of their taxes, they would just provide the certificate showing the taxes to be in order, together with the proof of the inclusion in the corresponding Merkle tree. Since we have no way to detect that the other certificate is also in the tree (due to the data being secret), we can not detect foul play in this case. A similar sort of embezzlement can be executed with contracts stating how much public funds were used to execute a particular activity, or spent on a particular service.

\domagoj{add some potential solutions}

