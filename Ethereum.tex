The idea of having many different organizations backing up the Merkle trees published by the government works well, however, it is still a solution suffering from some weaknesses. For instance, when there are very few such organizations, or individuals willing to guard (a portion of) the security certificates needed to verify the correctness of the data, they could be easily subverted, or give little additional credibility. Overall, one might argue that this solution is easily susceptible to corruption, since the participating organizations do not have any (material) gain from doing their duty correctly. Namely, if a malicious government agent could bribe the participating organizations to go along and change the data as needed, pretending that it was correct from the start. %Notice that individuals who guarded (a part of) the security certificates, would have even less power in this case, since there is more trust in established organizations, than in individuals.


The solution to this problem is quite simple: publish the Merkle root (together with the hash of the previous day's data) announced by the government on a well established public blokchain. In case that the blockhcain used is sufficiently economically powerful, as is the case for in stance with Bitcoin or Ethereum these days, it is highly unlikely that a malicious agent can subvert the entire blockchain to change the data published once. In this solution, the government would additionally to publishing the data, a listing of the data, and the Merkle tree of the information published for a single day, also commit a {\em single} transaction to e.g. Ethereum's blockchain signed by the secret key corresponding to the government's public key, and containing the hash value of the Merkle root for that day (hashed together with the hash value of previous day's data).

This amounts to the most lightweight solution proposed in Section \ref{sec:basic}, where only the root is guarded. While this does not allow us to verify which particular document (if any) was modified, it does allow us to check whether a day's data was manipulated. The restriction that a single transaction is published per day is there to ensure that  the government commits to a particular view of the data for a single day. This way, it is straightforward to locate the entire history on the blockchain by looking for transactions corresponding to the government's public key, and guarding the hash values these transactions commit to. Notice that this solution is pretty lightweight for the blockchain it is published on since it commits a single transaction per day, and can also be implemented in a fairly economical way.

This can be further implemented using a smart contract ME IMAGINO QUE SI??? ESCRIBAN DETALLES DE STEALA HASH AQUI.