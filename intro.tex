The philosophy behind the Open Government Data initiative is based on the idea that making the government data freely available to the general public will make the democratic process more transparent, and the government institutions accountable for their actions. There is already a large number of governments that have committed to opening up their data, for instance \url{data.gov} in the US, \url{data.gov.uk} in the UK, and \url{data.gov.in} in India, to name a few. International institutions such as The Organisation for Economic Co-operation and Development (OECD) are also openly promoting the adoption of the open data policy\footnote{\url{http://www.oecd.org/gov/digital-government/open-government-data.htm}}.

Despite the increasing enthusiasm around the initiative, there are also sceptic voices that illustrate some vulnerabilities of systems for publishing government data. For instance, a recent report by the Economist Intelligence Unit \cite{economist} found that there is a non negligible percentage of the population concerned by the governments' ability to keep the data secure from a modification by third parties. Similarly, there is a raising concern that the government might be manipulating historical data or erasing records \cite{poynter}.

In both of these examples, it might be difficult even for an investigative journalist with significant institutional support to prove foul play with regard to government records. First of all, they would have to keep a copy of the government data. Given the volume of the data that the government publishes, this can be prohibitively expensive. Second, even if a person had an (alleged) local copy of the data, how are we to know whether they modified this copy, or whether the government, or a third party did it with the data that is currently available? In any case, we would need to assign an arbiter to decide the dispute, which leads to similar trust issues.


%After all, to challenge the veracity of the data that is publicly available, one usually uses a local copy they made previously, showing it to be different to the actual data\footnote{In practice one might use a more elaborate mechanism that includes web records or cached web pages, but our point still remains valid.}. But how are we to know whether the challenger modified his the copy of the data, or whether the government, or a third party did it? In any case, we would have to assign an arbiter to decide the dispute, thus relying on one single institution (a judge or a tribunal most likely), which can lead to similar trust issues.


Blockchain technology allows us to remove both of these issues on a technological level. The basic solution is quite simple, and does not even require a major disruption to the way that the government publishes its data. To implement this solution, we rely on the blockchain data structure underlying Bitcoin \cite{whitepaper,bitcoinbook}. In essence, a blockchain is a list where each element stores not only its data, but also a cryptographic hash of the previous element of the list \cite{bitcoinbook}. The government can then publish its data in this format, signing each element of the list (also called a block\footnote{From now on will use the term blockchain and blockchain interchangeably. Note that this is not a standardized convention in the literature, as the term blockchain often includes one or more elements of the distributed consensus protocol used to create and manage the underlying hash-list \cite{NarayananC17}.}) with its private key in order to confirm the identity of the publisher. Note that this solution is basically identical to simple centralized cryptocurrencies such as Scroogecoin \cite{bitcoinbook}.

By the properties of the hash function, the only piece of information we need to store in order to assure that the data has not been modified up to some specific element of the list is the hash value of that element. Because of this, storing the government data in a hash-list allows us to resolve the two issues discussed above. First, to make sure the data has not been tampered with, one does not need to store the data itself, but only store a hash value of the latest block. Even if one stores this information frequently, it will still be small enough to be handled even by smartphones. Second, since storing the information required to assure that the data has not been manipulated is so cheap, multiple copies can be made available from trusted sources (e.g. non profit organizations, public activists, concerned citizens, etc.), thus making it infeasible to manipulate all of these sources by a malicious agent.



%While the term blockchain is often used to describe Bitcoin's public ledger and various elements used in the decentralization protocol involved in deciding what data will be written on this ledger (see \cite{NarayananC17} for a good overview), here we will use the term blockchain to refer to simple hash-list that assures we can detect when the data in some previous block has been modified. We  therefore think of a blockchain as a list where each element stores not only its data, but also a cryptographic hash of the previous element of the list \cite{bitcoinbook}. By the properties of the hash function, the only piece of the information we need to store in order to assure that the data has not been modified up to some specific element of the list the hash value of that element. 



%Storing the government data on a blockchain allows us to resolve the two issues discussed above. First, in order to assure that the data has not been modified we do not need to keep a copy of the data itself, but only of its cryptographic hash \cite{bitcoinbook}, which can be done even on a modern phone, and does not require a prohibitive amount of storage. Second, since storing the information required to assure that the data has not been manipulated is so cheap, multiple copies can be made available from trusted sources (e.g. non profit organizations, public activists, etc.), thus making it infeasible to manipulate all of these sources by a malicious agent.

%While it is certainly feasible to develop a simple blockchain for publishing government data (basically implementing a Scroogecoin based solution described in \cite{bitcoinbook}), and achieve the guarantees sought in the scenarios we just described, this solution might run into some issues when managing government data. More precisely, we would like to tackle the following problems:

While this simple solution does work in resolving the two issues outlined above, due to the way that government data is managed and published, applying it ``out of the box" might not be feasible. Next, we list several technical challenges that need to be resolved in order to make it possible to publish government data in a hash-list (or any other blockchain-based platform):
\begin{itemize}
\item {\it Credibility.} The mistrust of the general public is not solved by simply publishing the data on a (centralized) blockchain. Most notably, if there is not enough interest by non-profit agencies or the general public to occasionally back up the latest hash (and possibly check the veracity of the information), the publisher can change the data, recompute the blockchain, and upload it again. %To resolve this, we therefore propose to occasionally publish hashes of the government's blocks on a well established decentralized blockchain such as e.g. Ethereum, and issue certificates to the current state of the governments' blockchain.
\item {\it Persistence.} By its nature, the records that the government keeps are often updated to fix typos or reflect new findings. On the other hand, the data on a blockchain is immutable once published. It is therefore important to design a blockchain-based platform for government data that allows updates to the information, and tracks them in an  immutable way.
\item {\it Tracking documents.}
%Since the volume of data that the government publishes might be large, it can be difficult to locate a specific document that one has at hand. In particular, the document's hash will probably be ``hidden" in the Merkle tree for the particular block that the document is in, thus making it difficult to find.
Given the relatively large number of documents published by governments and the fact that storing records in the blockchain is expensive, it is not reasonable to store a different identifier for each document in the blockchain. Instead, one could store enough information to allow for irrefutable timestamped proofs of existence for documents using less space in the blockchain (e.g. use a Merkle tree and only store its root in the blockchain). Unfortunately, this introduces the need to have additional information for tracking a document, as the document itself is not enough to produce an identifier that can be verified in the blockchain.
\francisco{I find this Merkle tree occurrence too son.}\martin{I re-wrote this part considering that the issue is not the large amount of data but the fact that the documents' hashes cannot be verified without a certificate}
%Furthermore, it might be difficult to verify if this particular document was ever ``committed" to a public backup of the data described in point one above, since the document's hash will probably be ``hidden" in the Merkle tree for the particular block it is in.

\item {\it Clairvoyance attacks.} Note that using a blockchain-based solution the government can commit to never changing its secret and sensitive data by publishing their hashes. However, it might be possible for a misbehaving government agent to publish various versions of the same document in the blockchain, and in the case of a dispute reveal only the one suiting him best. Since the data itself is published through cryptographic hashes, it will be impossible to differentiate such versions.
%\item {\it Infrastructure issues.} Finally, we have to address the way that government data is created and stored, and the way that particular hash functions can be made vulnerable is not used properly. 
\end{itemize}

In this paper we describe the components of a platform for publishing government data that aims at resolving the technical challenges outlined above. In particular, 
we resolve the credibility issue by publishing occasional hashes of the latest blocks on a publish blockchain supporting smart contracts (in our case Ethereum, but the solution can be extended to many similar platforms). This will also allow us to track documents in a similar way. On the other hand, the persistence issue is solved by designing a specialized protocol for publishing the data that issues a smart contract based certificates for tracking all the updates to a document and to confirm its most recent version. Finally, the problem of clairvoyance attacks is discussed, and we show how clairvoyance attacks be avoided by sacrificing the flexibility of the type of documents that can be stored by the system.

%\francisco{I am a bit skeptic about this phrase. I would rather say that clairvoyance attacks can be avoided, but at the expense of flexibility (you have to impose some format of the target documents, like json objects as in \cite{energiaabierta})}

We would like to note that this paper is certainly not the first one to consider blockhain-based technologies for managing government data. Indeed, this idea has been around for a while, and ranges from fully fledged public blockchains recording every action of the government \cite{eushit,ibmshit}, to smaller scale solutions that track a fixed series of data items \cite{energiaabierta}. What we believe makes our solution technically interesting is the idea of allowing the data to be updated and secured using smart contracts. We also do not provide only a conceptual description of our solution, but also specify the components needed to make it work on a concrete platform (Ethereum in our case). We also discuss some technical issues that have to be handled in this specific application scenario.

In what follows, we start by recapping the basic notions about hash functions, public signatures, and providing a simple protocol for publishing government data using a hash-list. We then move onto providing solutions to each of the technical issues described above.
