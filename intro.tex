The philosophy behind Open Government Data initiative is based on the idea that making the government data freely available to the general public will make the democratic process more transparent, and the government institutions accountable for their actions. There is already a large number of governments that have committed to opening up their data (for instance \url{data.gov} in the US, \url{data.gov.uk} in the UK, and \url{data.gov.in} in India, to name a few), and international institutions such as The Organisation for Economic Co-operation and Development (OECD) are openly promoting the adoption of the open data policy\footnote{\url{http://www.oecd.org/gov/digital-government/open-government-data.htm}}.

Despite the fact that there is a lot of enthusiasm around the initiative, there are also sceptic voices that illustrate some vulnerabilities of  systems for publishing government data. For instance, a recent report by the Economist Intelligence Unit \cite{economist} found that there is a non negligible percentage of the population concerned by the governments' ability to keep the data secure from a modification by third parties. Similarly, there is a raising concern that the government itself might be manipulating the data or erasing records \cite{poynter}. 

In both of these examples, it might be difficult for a member of the general public (or even an investigative journalist with significant institutional support) to prove foul play with regards to government records. First of all, they would have to keep a copy of the government data. Given the volume of the data that the government publishes, this can be prohibitively expensive. Second, even if a person had an (alleged) local copy of the data, how are we to know whether they modified this  copy, or whether the government, or a third party did it with the data that is currently available? In any case, we would have to assign an arbiter to decide the dispute, thus relying on one single institution (a judge or a tribunal most likely), which can lead to similar trust issues.


%After all, to challenge the veracity of the data that is publicly available, one usually uses a local copy they made previously, showing it to be different to the actual data\footnote{In practice one might use a more elaborate mechanism that includes web records or cached web pages, but our point still remains valid.}. But how are we to know whether the challenger modified his the copy of the data, or whether the government, or a third party did it? In any case, we would have to assign an arbiter to decide the dispute, thus relying on one single institution (a judge or a tribunal most likely), which can lead to similar trust issues.


The good news is that the blockchain technology allows us to remove both of these issues on a technological level. While the term blockchain is often used to describe Bitcoin's public ledger and various elements used in the decentralization protocol involved in deciding what data will be written on this ledger (see \cite{NarayananC17} for a good overview), here we will use the term blockchain to refer to simple hash-list that assures we can detect when the data in some previous block has been modified. We  therefore think of a blockchain as a list where each element stores not only its data, but also a cryptographic hash of the previous element of the list \cite{bitcoinbook}. By the properties of the hash function, the only piece of the information we need to store in order to assure that the data has not been modified up to some specific element of the list the hash value of that element. 



EXPLAIN THE SOLUTION IN A PARAGRAPH. SAY HOW IT SOLVES THE ISSUES. SAY WHAT WE DO.


Storing the government data on a blockchain allows us to resolve the two issues discussed above. First, in order to assure that the data has not been modified we do not need to keep a copy of the data itself, but only of its cryptographic hash \cite{bitcoinbook}, which can be done even on a modern phone, and does not require a prohibitive amount of storage. Second, since storing the information required to assure that the data has not been manipulated is so cheap, multiple copies can be made available from trusted sources (e.g. non profit organizations, public activists, etc.), thus making it infeasible to manipulate all of these sources by a malicious agent.

%While it is certainly feasible to develop a simple blockchain for publishing government data (basically implementing a Scroogecoin based solution described in \cite{bitcoinbook}), and achieve the guarantees sought in the scenarios we just described, this solution might run into some issues when managing government data. More precisely, we would like to tackle the following problems:

While this simple description 

\begin{itemize}
\item {\it Credibility.} The mistrust of the general public is not solved by simply publishing the data on a (centralized) blockchain. Namely, the publisher can change the data, recompute the blockchain, and upload it again.
\item {\it Persistence.} By its nature, the records that the government keeps are often updated to fix typos or reflect new findings. On the other hand, the data on a blockchain is immutable once published.
\item {\it Erasing documents.} Since the volume of data that the government publishes is large, it can be difficult to locate a specific document, thus making it easy for the government to make this document disappear.
\item {\it "Clairvoyance attacks".} If sensitive data is published only through hashes, it might be possible for a misbehaving government agent to publish various versions of a document in the same block, and in the case of a dispute reveal only the one suiting him best.
\item {\it Infrastructure issues.} Finally, we have to address the way that government data is created and stored, and the way that particular hash functions can be made vulnerable is not used properly. 
\end{itemize}

