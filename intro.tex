The philosophy behind Open Government Data initiative is based on the idea that making the government data freely available to the general public will make the democratic process more transparent, and the government institutions accountable for their actions. There is already a large number of governments that have committed to opening up their data (for instance \url{data.gov} in the US, \url{data.gov.uk} in the UK, and \url{data.gov.in} in India, to name a few), and international institutions such as The Organisation for Economic Co-operation and Development (OECD) are openly promoting the adoption of the open data policy\footnote{\url{http://www.oecd.org/gov/digital-government/open-government-data.htm}}.

Despite the fact that there is a lot of enthusiasm around the initiative, there are also some sceptic voices that illustrate some vulnerabilities of  systems for publishing government data. For instance, a recent report by the Economist Intelligence Unit \cite{economist} found that there is a non negligible percentage of the population concerned by the governments' ability to keep the data secure from a modification by third parties. Similarly, there is a raising concern that the government itself might be manipulating the data or erasing records \cite{poynter}. 

In both of these examples, it might be difficult for a member of the general public (or even an investigative journalist with significant institutional support) to prove foul play with regards to government records. After all, to challenge the veracity of the data that is publicly available, one usually uses a local copy they made previously, showing it to be different to the actual data\footnote{In practice one might use a more elaborate mechanism that includes web records or cached web pages, but our point still remains valid.}. But how are we to know whether the challenger did modify the copy of the data himself, or whether the government, or a third party did it? In any case, we would have to assign an arbiter to decide the dispute, thus relying on one single institution (a judge or a tribunal most likely).


Explain the setting and the objective of the platform.

List technical issues and explain how we propose to tackle them (sections below) -- this is what the paper is about.

\begin{itemize}
\item {\it Credibility.} The mistrust of the general public is not solved by simply publishing the data on a (centralized) blockchain. Namely, the publisher can change the data, recompute the blockchain, and upload it again.
\item {\it Persistence.} By its nature, the records that the government keeps are often updated to fix typos or reflect new findings. On the other hand, the data on a blockchain is immutable once published.
\item {\it Erasing documents.} Since the volume of data that the government publishes is large, it can be difficult to locate a specific document, thus making it easy for the government to make this document disappear.
\item {\it "Clairvoyance attacks".} If sensitive data is published only through hashes, it might be possible for a misbehaving government agent to publish various versions of a document in the same block, and in the case of a dispute reveal only the one suiting him best.
\item {\it Infrastructure issues.} Finally, we have to address the way that government data is created and stored, and the way that particular hash functions can be made vulnerable is not used properly. 
\end{itemize}