


Explain the setting and the objective of the platform.

List technical issues and explain how we propose to tackle them (sections below) -- this is what the paper is about.

\begin{itemize}
\item {\it Credibility.} The mistrust of the general public is not solved by simply publishing the data on a (centralized) blockchain. Namely, the publisher can change the data, recompute the blockchain, and upload it again.
\item {\it Persistence.} By its nature, the records that the government keeps are often updated to fix typos or reflect new findings. On the other hand, the data on a blockchain is immutable once published.
\item {\it Erasing documents.} Since the volume of data that the government publishes is large, it can be difficult to locate a specific document, thus making it easy for the government to make this document disappear.
\item {\it "Clairvoyance attacks".} If sensitive data is published only through hashes, it might be possible for a misbehaving government agent to publish various versions of a document in the same block, and in the case of a dispute reveal only the one suiting him best.
\item {\it Infrastructure issues.} Finally, we have to address the way that government data is created and stored, and the way that particular hash functions can be made vulnerable is not used properly. 
\end{itemize}