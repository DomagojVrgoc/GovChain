% !TEX root = GovChain.tex


While the basic solution we proposed goes in the correct direction, it still suffers from some deficiencies which can be addressed (at least partially) on a technological level. In this section we discuss how to further increase the credibility of this way of publishing data, how to deal with updates to the data and checking currently valid versions, as well as how to deal with issues arising when the government wants to guaranty the immutability of their secret or sensitive data. \francisco{Didn't understand this last point. }

\subsection{Using an established blockchain to achieve credibility.}
% !TEX root = GovChain.tex

The idea of having many different organizations backing up the Merkle trees published by the government works well, however it is still a solution suffering from some weaknesses. For instance, when there are very few such organizations, or individuals willing to guard (a portion of) the security certificates needed to verify the correctness of the data, they could be easily subverted, or give little additional credibility. Overall, one might argue that this solution is easily susceptible to corruption, since the participating organizations do not have any (material) gain from doing their duty correctly. Moreover, a malicious government agent might be able to bribe the participating organizations or just change the data as needed and persecute anyone trying to disclose the truth.
%to change the (Merkle tree and blockchain) data as needed, pretending that it was correct from the start. %Notice that individuals who guarded (a part of) the security certificates, would have even less power in this case, since there is more trust in established organizations, than in individuals.

The solution to this problem is quite simple: publish the top level blockchain from Figure \ref{fig-merkle-chain} announced by the government on a well established public blokchain. That is, in addition to the government and the monitoring agencies publishing this data on their web pages, each day's block is sent to a public blockchain. In case that the blockhcain used is sufficiently secure, as Bitcoin or Ethereum are these days, it is practically unfeasible to change the data once it has been published. In this solution the government, in addition to publishing the actual data, also commits a {\em single} signed transaction to e.g. Ethereum's blockchain, backing up the blockchain of Merkle roots (see Figure \ref{fig-merkle-blockchain}).
% containing the hash value of the Merkle root, hashed together with the hash value of previous day's data.

This amounts to the most lightweight solution proposed in Section \ref{sec:basic}, where only the root is monitored. While this does not allow to trace modifications of single documents, it does allow to verify proofs for valid documents. In other words, one can generate a certificate to convince anyone that a document was indeed published on a particular day. The restriction that a single transaction is published per day is there to ensure that the government commits to a particular view of the data for that day. This way, it is straightforward to locate the entire history on the blockchain by looking at the information stored in the corresponding smart contract, and guarding the hash values these transactions commit to. This solution is extremely lightweight for the Ethereum network (which makes it very cheap) while allowing to audit a government publishing an arbitrarily large number of documents per day.

\begin{figure}
%\label{fig-merkle-blockchain}
\centering
\begin {tikzpicture}[-latex ,auto ,node distance =1 cm and 2.5 cm ,on grid, semithick, state/.style ={ rectangle , draw, minimum width =0.9 cm}, substate/.style ={ rectangle , draw, minimum width =0.2 cm}]

\usetikzlibrary{shapes}
\node (x1) {};
\node (x2) [right = 2.7cm of x1]{};
\node (x3) [right = 2.7cm of x2]{};
\node (etc) [right = 1.8cm of x3]{};

\node[state] (r1) [below =of x1]{\scriptsize$\mathit{MR}_1$};
\draw [draw=black!50] ($(r1.south west) - (0.4,0.1cm)$) rectangle +(1.75cm, +1.2cm);
\node (h1) [above =15pt of r1]{$\bot$};
\node (bl1) [above =30pt of r1]{$B_1$};

\node[state] (r2) [right = 2.7 cm of r1]{\scriptsize$\mathit{MR}_2$};
\draw [draw=black!50] ($(r2.south west) - (0.4,0.1cm)$) rectangle +(1.75cm, +1.2cm);
\node (h2) [above =15pt of r2]{$h(B_1)$};
\node (bl2) [above =30pt of r2]{$B_2$};

\node[style={ellipse}] (BC) [above =2.5cm  of r3] {Ethereum};

\path (bl1) edge[dotted, bend left=25] node {} (BC);
\path (bl2) edge[dotted, bend left=25] node {} (BC);
\path (bl3) edge[dotted, bend right=25] node {} (BC);

\node[state] (r3) [right = 2.7cm of r2]{\scriptsize$\mathit{MR}_3$};
\draw [draw=black!50] ($(r3.south west) - (0.4,0.1cm)$) rectangle +(1.75cm, +1.2cm);
\node (h3) [above =15pt of r3]{$h(B_2)$};
\node (bl3) [above =30pt of r3]{$B_3$};

% First childs

\node[substate] (c11) [below left  = 0.9 of r1]{};
\node[substate] (c12) [below right = 0.9 of r1]{};
\node[substate] (c21) [below left  = 0.9 of r2]{};
\node[substate] (c22) [below right = 0.9 of r2]{};
\node[substate] (c31) [below left  = 0.9 of r3]{};
\node[substate] (c32) [below right = 0.9 of r3]{};

% Second childs

\node[substate] (c111) [below left  = 0.6 and 0.2 of c11]{};
\node[substate] (c112) [below right = 0.6 and 0.2 of c11]{};
\node[substate] (c121) [below left  = 0.6 and 0.2 of c12]{};
\node[substate] (c122) [below right = 0.6 and 0.2 of c12]{};

\node[substate] (c211) [below left  = 0.6 and 0.2 of c21]{};
\node[substate] (c212) [below right = 0.6 and 0.2 of c21]{};
\node[substate] (c221) [below left  = 0.6 and 0.2 of c22]{};
\node[substate] (c222) [below right = 0.6 and 0.2 of c22]{};

\node[substate] (c311) [below left  = 0.6 and 0.2 of c31]{};
\node[substate] (c312) [below right = 0.6 and 0.2 of c31]{};
\node[substate] (c321) [below left  = 0.6 and 0.2 of c32]{};
\node[substate] (c322) [below right = 0.6 and 0.2 of c32]{};

\node (etc111) [below = 0.3 of c111]{$\vdots$};
\node (etc112) [below = 0.3 of c112]{$\vdots$};

\node (etc121) [below = 0.3 of c121]{$\vdots$};
\node (etc122) [below = 0.3 of c122]{$\vdots$};

\node (etc21) [below = 0.9 of c21]{$\vdots$};
\node (etc22) [below = 0.9 of c22]{$\vdots$};
\node (etc31) [below = 0.9 of c31]{$\vdots$};
\node (etc32) [below = 0.9 of c32]{$\vdots$};

% Documents

\node (ghost1) [below left = 1.2 and 0.3 of c11]{};
\node (ghost2) [right = 1 of ghost1]{};
\node (ghost3) [right = 1 of ghost2]{};

\node (d1) [below left  = 0.8 and 0.3 of ghost1]{$d_1$};
\node (d2) [right = 0.5 of d1]{$d_2$};
\node (d3) [below left  = 0.8 and 0.25 of ghost2]{$d_3$};
\node (d4) [right = 0.5 of d3]{$d_4$};
\node (d5) [right = 0.5 of d4]{...};
\node (d6) [right = 0.5 of d5]{$d_k$};

%\path (r1) edge node {} (x1);
%\path (r2) edge node {} (x2);
%\path (r3) edge node {} (x3);
%\path (r3) edge node {} (x3);

%\path (x1) edge node {} (x2);
%\path (x2) edge node {} (x3);
%\path (x3) edge node {} (etc);

\path (r1) edge node {} (c11);
\path (r1) edge node {} (c12);
\path (r2) edge node {} (c21);
\path (r2) edge node {} (c22);
\path (r3) edge node {} (c31);
\path (r3) edge node {} (c32);

\path (c11) edge node {} (c111);
\path (c11) edge node {} (c112);
\path (c12) edge node {} (c121);
\path (c12) edge node {} (c122);

\path (c21) edge node {} (c211);
\path (c21) edge node {} (c212);
\path (c22) edge node {} (c221);
\path (c22) edge node {} (c222);

\path (c31) edge node {} (c311);
\path (c31) edge node {} (c312);
\path (c32) edge node {} (c321);
\path (c32) edge node {} (c322);

\path (ghost1) edge node {} (d1);
\path (ghost1) edge node {} (d2);
\path (ghost2) edge node {} (d3);
\path (ghost2) edge node {} (d4);
\path (ghost3) edge node {} (d6);

\node (ar1) [above right = 10pt and 0.77cm of r1] {};
\node (ar2) [above right = 10pt and 1.95cm of r1] {};
\node (ar3) [above right = 10pt and 3.47cm of r1] {};
\node (ar4) [above right = 10pt and 4.65cm of r1] {};

\path (ar2) edge node {} (ar1);
\path (ar4) edge node {} (ar3);

\end{tikzpicture}
\caption{In addition to the broadcast of single documents and the Merkle root, the blockchain $[B_1,B_2,\dots]$ is published element-wise to a well established blockchain. Anyone having access to a single path in a tree and the previous day hash can verify the integrity of the whole tree.}
\label{fig-merkle-blockchain}
\end{figure}

\lstset{basicstyle=\footnotesize}
\lstset{
  numbers=left,
  stepnumber=1,    
  firstnumber=1,
  numberfirstline=true
}
\begin{figure*}
\begin{lstlisting}[language=java]
pragma solidity ^0.4.24;

contract GovChain{

    mapping (bytes32 => bytes32) public roots;
    mapping (bytes32 => bytes32) public blocks_chain;
    mapping (bytes32 => uint) public timestamps;
    bytes32 public last_block_hash;
    address public publisher_address;

    function GovChain() public {
        publisher_address = msg.sender;
        last_block_hash = 1;
    }

    function add_root(bytes32 merkle_root) public {
        // Check that the sender is correct
        assert(msg.sender == publisher_address);
        // Check the elapsed time is at least 20 hours
        assert(block.timestamp - timestamps[last_block_hash] > 72000);        
        // Hash the concatenation of the previous block's hash and the new Merkle root
        bytes32 new_block_hash = sha256(last_block_hash, merkle_root);
        // Check that this hash has not been uploaded before.
        assert(roots[new_block_hash] == 0);
        roots[new_block_hash] = merkle_root;
        blocks_chain[new_block_hash] = last_block_hash;
        timestamps[new_block_hash] = block.timestamp;
        last_block_hash = new_block_hash;
    }
}

\end{lstlisting}
\caption{Solidity code for backing up the top level blocks of Figure \ref{fig-merkle-chain} on Ethereum's blockchain. Note that the code itself constructs the blockchain, thus giving additional credibility to the blockchain published by the government.}
\label{contract}
\end{figure*}

To make things concrete, we next illustrate how this can be achieved in Ethereum. In Figure \ref{contract} we give a Solidity contract for publishing the blockchain of Merkle roots in a secure way. Solidity \cite{Solidity} is a high level language for describing smart contracts that is then compiled to be executed on Ethereum's virtual machine. The contract we present is to be called by the government publishing the data, and they will be the only ones able to execute the contract (assuming their private key does not get compromized). This is assured by guarding the contract publisher's address (in form of their public key) when the contract is deployed (line 12 of Figure~\ref{contract}), and then checking that it is this particular address calling the contract to upload new values (line 18). The contract itself guards the data corresponding to the block of each day, plus the time they were published. For this we maintain a mapping (i.e. a dictionary) \texttt{roots} linking the hash of each block to the corresponding Merkle root, a mapping \texttt{blocks\_chain} linking the hash of each block to the hash of the previous block of the blockchain (e.g. $h(B_2)$ in $B_3$ of Figure \ref{fig-merkle-chain}), as well as a mapping \texttt{timestamps} that tells us the time each block was published. We also store the hash value of the last block in the variable \texttt{last\_block\_hash}.

The only function that the contract can execute once deployed is for extending the list by a new Merkle root. This is achieved by the function \texttt{add\_root}, which takes the Merkle root value as the input and adds it to the blockchain. Note that our contract allows the blocks to be submitted at most once a day (in line 20 we require the new block to have a delay of at least 20 hours to allow some flexibility on when the day's data is published). If these verifications complete successfully, we then compute the hash value of the new element of our list (line 22), check it was not used before (line 24), and define its root value as the hash value received as input to the function (line 25). We then also set the value of the previous element of the list in line 26, the timestamp in line 27, and update the value of the ultimate element of the list as this one in line 28.

Note that this contract can also serve as the verification that the blockchain published by the government is indeed the correct one, as it recomputes its hash values (line 22 of the contract). This way, we have additional assurance that the government's data about the blockchain is correct, since it is recomputed on Ethereum's blockchain, in addition to being published elsewhere.

Overall, publishing the data on Ethereum's blockchain using a contract presented in Figure~\ref{contract} is a very efficient and lightweight solution. In particular, since submitting Merkle roots is limited to one daily, the cost of this contract will be very low (way below one dollar per day at the current Ethereum/dollar exchange rate), and does not heavily contribute to the congestion of Ethereum's network. Furthermore, having the value of any element of the published blockchain allows us to access it for free by connecting to Ethereum nodes (or running one), and can provide us with the value of that day's Merkle root, date, and the hash value of the previous element of the list. This in turn allows us to obtain data directly from the government and, plus a succinct certificate that should also be provided by the government, verify that we are obtaining the correct information.

%Finally, notice that the idea of monitoring organizations presented in Section \ref{sec:basic} can now be further simplified, as they do not need to keep the entire top level blockchain of Merkle roots any more, but can just occasionally download the hash value of the latest block on this list. Since the list itself is being guarded in the contract on the Ethereum's blockchain, looking up this value allows us to both check that it corresponds to the correct day's data (through the value of the mapping \texttt{timestamps}), that it matches the corresponding Merkle root, and includes the correct hash of the previous day's data. In essence, by committing to this smart contract, the government can not go back and change it's day digest in any way, and there can be no dispute on whether the Merkle root of the day's data provided by the government and the one in possession of a monitoring agency/person is different, modulo the security of Ethereum's blockchain.
%\martin{I highly disagree here. The use of monitoring organizations is now useless for storing what is already stored in the smart contract. However, it is still (very) relevant for on-line verification that the government is doing everything correctly and for storing all of the data in case the government refutes to provide a certificate. Also, these organizations should provide anti-certificates in case the government says something incorrect.}

Finally, notice that the idea of monitoring organizations presented in Section \ref{sec:basic} can now be further strengthened. First, by committing to this smart contract, the government can not go back and change it's day digest in any way, because if it does this, the monitoring agency can easily pinpoint the change based on the data available in the smart contract and the one they store locally. Second, if the government refuses to provide a certificate for a specific document that was originally present, a monitoring agency that also store the Merkle trees for each day (or has a path to the root of the tree where this document was stored) can easily call them out for this and show that the document was indeed present in the government's data for that day, thus making it impossible for documents to disappear.


%\domagoj{This can be further implemented using a smart contract ME IMAGINO QUE SI??? ESCRIBAN DETALLES DE STEALA HASH AQUI.}
%
%\domagoj{IF MONITORING ORGANIZATIONS ARE WELL ESTABLISHED MAYBE MOVE TO A N OUT OF M SIGNATURE TO COMMIT/OR AN ADDITIONAL TRANSACTION THEY PUBLISH WITH N OUT OF M SIGNATURES TO VERIFY THE DAY'S DATA?}\francisco{Or they could actually sign the same contract as the government}
%
%In fact, the only blockchain-related technological need for this is a tool that lets an authorized party to daily publish one short string in a respected blockchain (typically, the signed output of a SHA-256 or BLAKE2 computation into Ethereum or Bitcoin blockchains). This is simple and of negligeable cost. We have witnessed some initiatives aiming to do this since the early years of Bitcoin, but we have yet to see one that is appropriate for government use. Some of them are vulnerable to the well-known length-extension attacks, most of them do not support cryptographic signatures on published hashes, and almost all of them charge the user a large, arbitrary amount for this relatively simple service. 
%
%\francisco{I think this section should include Martin's guarants or guards. Or maybe a separate section on it would be better.}



\subsection{Allowing the data to be updated.}
% !TEX root = GovChain.tex

It seems counterintuitive at first to use an established, immutable blockchain in order to commit to data that is subject to change often. It is yet a principal requirement of governments for data to be updated, given the complex architecture of hierarchies and the nature of the information the documents may contain. Documents collecting signatures may change daily, spreadsheets may be corrected for errors, and agendas of public projects may be filled with more details as the time passes. The basic solution given above with a small tweak actually allows to perform this.

The solution is based on the following simple idea: there are two types of leafs in our Merkle trees, new documents and new versions of previous documents. Whenever a new version of a previous document is published, it must include the hash value of the \emph{last version} of that same document, thus creating a blockchain structure for the history of each document. Now, since this is not occurring in a smart contract, the government could simply publish two different updates of the same document, thus \emph{branching} a document's history. However, in such a case an interested organization verifying the governments actions would have a trivial proof (verifyiable by anyone) showing that the government is doing something incorrect.

More formally, leafs of a Merkle tree would now look like $h_1,\ldots,h_j,(h_{j+1}:h_{j+1}'), \ldots,(h_n:h_n')$ where $h_1,\ldots,h_j$ are hashes of new documents and $(h_{j+1}:h_{j+1}'), \ldots,(h_n:h_n')$ represent that the document with hash $h_{i}$ is an updated version of the document with hash $h_{i}'$. Note again that there is not enough information in this data structure to verify that the government is behaving correctly. For example, we cannot immediately deduce that $h_{j+1}'$ is indeed the last version of a document (or if was ever published). However, an interested party verifying the publication of documents by the government would easily realize incorrect actions, and moreover would have a succinct and irrefutable proof of the government's errors.




\subsection{Impossible to erase documents.}
\input{erasing}


\subsection{Sensitive data and clairvoyance attacks.}
% !TEX root = GovChain.tex


Using the hiding property of a hash function, we might argue that the government might also guarantee for the immutability of their secret data that they do not want to publish. More precisely, assume that the government has a list $d_1,d_2,\ldots ,d_k$ of documents they wish to keep secret, but want to assure the general public that the documents themselves will not be changed not manipulated. For this, the government could simply create the Merkle tree $MT(d_1,\ldots ,d_k)$ that contains the hash values $h(d_1),\ldots ,h(d_k)$ as its leaves, and publish this tree as the safety certificate. If at a later date there is a dispute regarding some issue pertaining to document $d_i$ (e.g. in case that $d_i$ is a contract under question), a judicial body might require the government agency to make this particular document public, and show its membership in the Merkle tree $MT(d_1,\ldots ,d_k)$. In case that the document was manipulated in the meantime, this can now be detected, as the leave hash will not match the path to the root of the tree. Note that this still allows the documents different from $d_i$ to remain secret.

One problem with this solution is that since the documents $d_1,\ldots ,d_k$ are not made public, a malicious government agent might publish two versions of the document: one that suits him and does not reflect signs of foul play, and another that holds the actual true. For example, suppose that the government does not disclose tax certificates of particular entities, but includes these in the Merkle tree. Assume also that a certificate for an entity $X$ was issued stating that the taxes are not in order, and was included in a Merkle tree. An authorized government agent is then bribed to also issue a certificate that the taxes are in order, together with the proof of the inclusion in a Merkle tree (note that this can even be the same tree containing the first certificate). If $X$ is at a later date asked to prove the status of their taxes, they would just provide the certificate showing the taxes to be in order, together with the proof of the inclusion in the corresponding Merkle tree. Since we have no way to detect that the other certificate is also in the tree (due to the data being secret), we can not detect foul play in this case. A similar sort of embezzlement can be executed with contracts stating how much public funds were used to execute a particular activity, or spent on a particular service.

This can be addressed at the Merkle tree construction level, demanding some behaviour from the government and adding a simple verifying procedure to the guarding entities, as follows. Allow the government to input hashes of undisclosed documents in the Merkle tree leafs, but for each secret document, the government must input a public document containing (a) the hash value of the corresponding secret document and (b) some reasonable description thereof. Let us refer to these documents as \textit{affidavits}. In simple terms, an affidavit of the example above must read \textit{``A secret document with hash value $\langle \mbox{hash}\rangle$ has been inserted into this tree. Description: Tax situation of entity X. $\langle\mbox{timestamp}\rangle$, $\langle\mbox{signature}\rangle$''}. 

In order to avoid any ambiguities, we suggest that each affidavit corresponds to exactly one secret document in the same Merkle tree, and that they should be treated as regular public documents. The guarding entities facing such a tree must perform an additional verification: For each hash whose document they cannot access, there must be a affidavit in the same tree containing that hash value, and a description. There is a delicate degree of liberty on what a reasonable description should look like. We suggest that such a description contains the maximum amount of public information, in order to be used later as a proof. For instance, it should allow to trace two contradictory documents (a guarding entity may allow users to search affidavits by description). Should the government give unrelated or incomplete descriptions, this would become evident when a court orders to reveal the underlying document. 

Adding this to previous considerations, Merkle trees could also include elements of the form $(z:a)$ where $z$ is the hash of an undisclosed document (indicated in the body of the publicly available affidavit), and $a$ is the hash value of the affidavit itself.
%
%$$
%\begin{array}{ll}
%\mbox{\it New: } &\{x_1,x_2,\dots,x_n\},\\
%\mbox{\it Updates: }& \{(y_1:y_1'), (y_2:y_2'),\dots, (y_m:y_m')\},\\
%\mbox{\it Undisclosed: }& \{z_1,z_2,\dots,z_k\}\\
%\mbox{\it Affidavits: }& \{a_1,a_2,\dots,a_k\}\\
%\end{array}
%$$
%

%Upon publication of a Merkle root by the government, fully responsible verifiers should download all corresponding data, compute the hash value of each new document, check that each affidavit contains a hash value, and that updates link to a non-updated version of a document. After doing this, they assemble a Merkle tree with all computed data: (i) hashes of new documents, (ii) hashes of updated documents (possibly linking to the hash of the previous version, as discussed in \ref{sec:updates}) and (iii) hash values stated in affidavits. Note that affidavits are treated as regular disclosable documents, whose hash value is therefore included in (i). To finish, this entity verifies that the Merkle root of this tree matches the last published Merkle root in the established blockchain. There are some relatively straightforward details in these verifications which we omit here. 


%\domagoj{add some potential solutions}




\subsection{Formatting and hash function issues.}
% !TEX root = GovChain.tex
	
Finally, we would like to discuss some minor issues that need to be addressed when deploying a platform for publishing government's data in an  open and transparent manner.

First, the solution we propose might make it somewhat difficult to track a particular document. That is, if one simply has a document at hand, and no additional meta data, verifying where this document comes from, in which Merkle tree can we find its hash, or whether this is the latest version of the document might be difficult. To make this simpler, the most elegant solution would be that the API for downloading particular documents published by the government also includes the meta information needed to find the security certificates for a particular document. Furthermore, to track the latest version of the document, we can now turn either to the government's API or to the monitoring agencies, as described in Section \ref{sec:updates}.

%That is, when downloading each document, one would also obtain the root of its Merkle tree, together with the witnessing path of the inclusion in this Merkle tree, plus the tree's address in the Ethereum blockchain. With this information we can now easily check that the document we have was included in the official data published by the government, plus find a proof of this on Ethereum's blokchain. Furthermore, to track the latest version of the document, we can now turn either to the government's API or to the monitoring agencies, as described in Section \ref{sec:updates}.

%Another issue we face when dealing with government records is that they are usually not stored in standardized formats, and most likely use spreadsheet software such as Microsoft Excel, or text processors such as Microsoft Word. One issue with these tools is that they will change the document's hash value every time they open a document, even when no visible changes have been made (i.e. no new character has been added/removed), mostly due to the meta data they store along with the document. A similar issue can occur based on which operating system is used when processing the document. In both cases, a person who downloaded the document might compute its hash, and obtain a value that is different that that when a hash of the downloaded document was computed. To resolve this, one needs to have a clearly specified format in which the government's documents will be published, and inform the people using it of potential issues. Some solutions here include publishing data in binary format, or as pdf.

Another issue we face when dealing with government records is that they are usually not stored in standardized formats, and most likely use spreadsheet software such as Microsoft Excel, or text processors such as Microsoft Word. One issue with these tools is that they will change the document's hash value every time they open a document, even when no visible changes have been made (i.e. no new character has been added/removed), due to the meta data they store. In this case, a person who downloaded the document might compute its hash, and obtain a value that is different than the hash of the downloaded document. To resolve this, one needs to have a clearly specified format in which the government's documents will be published, and inform the people using it of potential issues. Some solutions here include publishing data in binary format, or as pdf.

Next, we also have to be careful with the specifics of the hash function being used. For instance, it is well known that popular choices such as SHA-256 is susceptible to length extension attacks \cite{lengthextension}, thus requiring to use it in a specific way (e.g. by using $h(x||h(x))$ instead of only $h(x)$). Alternatively, one can opt for more secure hash options that have been proved secure against such attacks \cite{keccak}.

Overall, these illustrate just some of the specific issues we will face when designing a platform for transparent publishing of government's data.

%	\francisco{Microsoft attacks, One millisecond attacks, Length extension attacks}

%	\francisco{Also the programmed updates attack (when documents are predictable and hash lists do not include outer data). Is it harmful?}


