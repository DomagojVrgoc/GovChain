\documentclass[conference]{IEEEtran}
\IEEEoverridecommandlockouts
% The preceding line is only needed to identify funding in the first footnote. If that is unneeded, please comment it out.
\usepackage{cite}
\usepackage{amsmath,amssymb,amsfonts}
\usepackage{algorithmic}
\usepackage{graphicx}
\usepackage{textcomp}
\usepackage{xcolor}
\def\BibTeX{{\rm B\kern-.05em{\sc i\kern-.025em b}\kern-.08em
    T\kern-.1667em\lower.7ex\hbox{E}\kern-.125emX}}
\begin{document}

\title{GovChain: a blockchain based platform for transparent publishing of government data\\
\thanks{Vial and Vrgo\v{c} were supported by the Nucleus Millennium Institute for Fundamentals of Data. Vrgo\v{c} is also supported by FONDECYT Grant Nr. 11160383.}
}

\author{\IEEEauthorblockN{Mart\'in Ugarte}
\IEEEauthorblockA{\textit{Universit\'e Libre de Bruxelles} \\
%\textit{name of organization (of Aff.)}\\
%City, Country \\
mugartec@ulb.ac.be}
\and
\IEEEauthorblockN{Francisco Vial}
\IEEEauthorblockA{\textit{PUC Chile and IMFD Chile} \\
%\textit{name of organization (of Aff.)}\\
%City, Country \\
fvial@uc.cl}
\and
\IEEEauthorblockN{Domagoj Vrgo\v{c} }
\IEEEauthorblockA{\textit{PUC Chile and IMFD Chile} \\
%\textit{name of organization (of Aff.)}\\
%City, Country \\
dvrgoc@ing.puc.cl}
}



\maketitle

\begin{abstract}
This document is a model and instructions for \LaTeX.
This and the IEEEtran.cls file define the components of your paper [title, text, heads, etc.]. *CRITICAL: Do Not Use Symbols, Special Characters, Footnotes, 
or Math in Paper Title or Abstract.
\end{abstract}

\begin{IEEEkeywords}
blockchain, government data, transparency
\end{IEEEkeywords}

\section{Introduction}

Explain the setting and the objective of the platform.

List technical issues and explain how we propose to tackle them (sections below) -- this is what the paper is about.

\begin{itemize}
\item {\it Credibility.} The mistrust of the general public is not solved by simply publishing the data on a (centralized) blockchain. Namely, the publisher can change the data, recompute the blockchain, and upload it again.
\item {\it Persistence.} By its nature, the records that the government keeps are often updated to fix typos or reflect new findings. On the other hand, the data on a blockchain is immutable once published.
\item {\it Erasing documents.} Since the volume of data that the government publishes is large, it can be difficult to locate a specific document, thus making it easy for the government to make this document disappear.
\item {\it "Clairvoyance attacks".} If sensitive data is published only through hashes, it might be possible for a misbehaving government agent to publish various versions of a document in the same block, and in the case of a dispute reveal only the one suiting him best.
\item {\it Infrastructure issues.} Finally, we have to address the way that government data is created and stored, and the way that particular hash functions can be made vulnerable is not used properly. 
\end{itemize}

\section{A toy solution}
Just the blockchain, not published on Ethereum

\section{Technical issues and how to address them}

\subsection{Using an established blockchain to achieve credibility.}

\subsection{Allowing the data to be updated.}

\subsection{Impossible to erase documents.}

\subsection{Sensitive data and clairvoyance attacks.}

\subsection{Formatting and hash function issues.}

Bla

\section{Potential impact and future directions}

\end{document}
