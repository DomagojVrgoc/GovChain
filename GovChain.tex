\documentclass[conference]{IEEEtran}
\IEEEoverridecommandlockouts
% The preceding line is only needed to identify funding in the first footnote. If that is unneeded, please comment it out.
\usepackage{cite}
\usepackage{amsmath,amssymb,amsfonts}
\usepackage{algorithmic}
\usepackage{graphicx}
\usepackage{textcomp}
\usepackage{xcolor}
\usepackage{url}
\usepackage{tikz}
\usetikzlibrary{positioning}

\def\BibTeX{{\rm B\kern-.05em{\sc i\kern-.025em b}\kern-.08em
    T\kern-.1667em\lower.7ex\hbox{E}\kern-.125emX}}
    
% Comments 

\newcommand{\martin}[1]{{\color{-orange!40!-olive}\textbf{Martín: #1}}}    
\newcommand{\domagoj}[1]{{\color{olive}\textbf{Domagoj: #1}}}
\newcommand{\francisco}[1]{{\color{orange!40!olive}\textbf{Francisco: #1}}}

% Crypto

\newcommand{\sk}{\mathit{sk}}
\newcommand{\pk}{\mathit{pk}}
\newcommand{\sign}{\mathit{sign}}
\newcommand{\verify}{\mathit{verify}}
\newcommand{\m}{\mathit{message}}

% Language

\newcommand{\ie}{i.e.$\,$}

    
    
    
\begin{document}

\title{GovChain: a blockchain based platform for transparent publishing of government data\\
\thanks{Vial and Vrgo\v{c} were supported by the Nucleus Millennium Institute for Fundamentals of Data. Vrgo\v{c} is also supported by FONDECYT Grant Nr. 11160383.}
}

\author{\IEEEauthorblockN{Mart\'in Ugarte}
\IEEEauthorblockA{\textit{Universit\'e Libre de Bruxelles} \\
%\textit{name of organization (of Aff.)}\\
%City, Country \\
mugartec@ulb.ac.be}
\and
\IEEEauthorblockN{Francisco Vial}
\IEEEauthorblockA{\textit{PUC Chile and IMFD Chile} \\
%\textit{name of organization (of Aff.)}\\
%City, Country \\
fovial@uc.cl}
\and
\IEEEauthorblockN{Domagoj Vrgo\v{c} }
\IEEEauthorblockA{\textit{PUC Chile and IMFD Chile} \\
%\textit{name of organization (of Aff.)}\\
%City, Country \\
dvrgoc@ing.puc.cl}
}



\maketitle

\begin{abstract}
Open government data is a valid effort on the part of leading world governments to make the democratic process more transparent. However, the sheer volume of the data makes it infeasible for a member of the general public to guard this information in order to assure that it is not manipulated by a malignant agent, or by the government itself.

In this short report we describe the main ideas behind a platform that enables government data to be verifiably secure and immutable. We start with a simple hash-list based structure to ensure immutability, peg its security to that of an existing blockchain enabling smart contracts, and then dwell into more delicate issues of how to deal with updating the government records, manage sensitive or secret data, and assure a simple way for a member of the general public to verify a history of a specific document.
\end{abstract}

\begin{IEEEkeywords}
blockchain, government data, transparency
\end{IEEEkeywords}

\section{Introduction}
\label{sec:intro}
The philosophy behind the Open Government Data initiative is based on the idea that making the government data freely available to the general public will make the democratic process more transparent, and the government institutions accountable for their actions. There is already a large number of governments that have committed to opening up their data, for instance \url{data.gov} in the US, \url{data.gov.uk} in the UK, and \url{data.gov.in} in India, to name a few. International institutions such as The Organisation for Economic Co-operation and Development (OECD) are also openly promoting the adoption of the open data policy\footnote{\url{http://www.oecd.org/gov/digital-government/open-government-data.htm}}.

Despite the increasing enthusiasm around the initiative, there are also sceptic voices that illustrate some vulnerabilities of systems for publishing government data. For instance, a recent report by the Economist Intelligence Unit \cite{economist} found that there is a non negligible percentage of the population concerned by the governments' ability to keep the data secure from a modification by third parties. Similarly, there is a raising concern that the government might be manipulating historical data or erasing records \cite{poynter}.

In both of these examples, it might be difficult even for an investigative journalist with significant institutional support to prove foul play with regard to government records. First of all, they would have to keep a copy of the government data. Given the volume of the data that the government publishes, this can be prohibitively expensive. Second, even if a person had an (alleged) local copy of the data, how are we to know whether they modified this copy, or whether the government, or a third party did it with the data that is currently available? In any case, we would need to assign an arbiter to decide the dispute, which leads to similar trust issues.


%After all, to challenge the veracity of the data that is publicly available, one usually uses a local copy they made previously, showing it to be different to the actual data\footnote{In practice one might use a more elaborate mechanism that includes web records or cached web pages, but our point still remains valid.}. But how are we to know whether the challenger modified his the copy of the data, or whether the government, or a third party did it? In any case, we would have to assign an arbiter to decide the dispute, thus relying on one single institution (a judge or a tribunal most likely), which can lead to similar trust issues.


Blockchain technology allows us to remove both of these issues on a technological level. The basic solution is quite simple, and does not even require a major disruption to the way that the government publishes its data. To implement this solution, we rely on the blockchain data structure underlying Bitcoin \cite{whitepaper,bitcoinbook}. In essence, a blockchain is a list where each element stores not only its data, but also a cryptographic hash of the previous element of the list \cite{bitcoinbook}. The government can then publish its data in this format, signing each element of the list (also called a block\footnote{From now on will use the term blockchain and blockchain interchangeably. Note that this is not a standardized convention in the literature, as the term blockchain often includes one or more elements of the distributed consensus protocol used to create and manage the underlying hash-list \cite{NarayananC17}.}) with its private key in order to confirm the identity of the publisher. Note that this solution is basically identical to simple centralized cryptocurrencies such as Scroogecoin \cite{bitcoinbook}.

By the properties of the hash function, the only piece of information we need to store in order to assure that the data has not been modified up to some specific element of the list is the hash value of that element. Because of this, storing the government data in a hash-list allows us to resolve the two issues discussed above. First, to make sure the data has not been tampered with, one does not need to store the data itself, but only store a hash value of the latest block. Even if one stores this information frequently, it will still be small enough to be handled even by smartphones. Second, since storing the information required to assure that the data has not been manipulated is so cheap, multiple copies can be made available from trusted sources (e.g. non profit organizations, public activists, concerned citizens, etc.), thus making it infeasible to manipulate all of these sources by a malicious agent.



%While the term blockchain is often used to describe Bitcoin's public ledger and various elements used in the decentralization protocol involved in deciding what data will be written on this ledger (see \cite{NarayananC17} for a good overview), here we will use the term blockchain to refer to simple hash-list that assures we can detect when the data in some previous block has been modified. We  therefore think of a blockchain as a list where each element stores not only its data, but also a cryptographic hash of the previous element of the list \cite{bitcoinbook}. By the properties of the hash function, the only piece of the information we need to store in order to assure that the data has not been modified up to some specific element of the list the hash value of that element. 



%Storing the government data on a blockchain allows us to resolve the two issues discussed above. First, in order to assure that the data has not been modified we do not need to keep a copy of the data itself, but only of its cryptographic hash \cite{bitcoinbook}, which can be done even on a modern phone, and does not require a prohibitive amount of storage. Second, since storing the information required to assure that the data has not been manipulated is so cheap, multiple copies can be made available from trusted sources (e.g. non profit organizations, public activists, etc.), thus making it infeasible to manipulate all of these sources by a malicious agent.

%While it is certainly feasible to develop a simple blockchain for publishing government data (basically implementing a Scroogecoin based solution described in \cite{bitcoinbook}), and achieve the guarantees sought in the scenarios we just described, this solution might run into some issues when managing government data. More precisely, we would like to tackle the following problems:

While this simple solution does work in resolving the two issues outlined above, due to the way that government data is managed and published, applying it ``out of the box" might not be feasible. Next, we list several technical challenges that need to be resolved in order to make it possible to publish government data in a hash-list (or any other blockchain-based platform):
\begin{itemize}
\item {\it Credibility.} The mistrust of the general public is not solved by simply publishing the data on a (centralized) blockchain. Most notably, if there is not enough interest by non-profit agencies or the general public to occasionally back up the latest hash (and possibly check the veracity of the information), the publisher can change the data, recompute the blockchain, and upload it again. %To resolve this, we therefore propose to occasionally publish hashes of the government's blocks on a well established decentralized blockchain such as e.g. Ethereum, and issue certificates to the current state of the governments' blockchain.
\item {\it Persistence.} By its nature, the records that the government keeps are often updated to fix typos or reflect new findings. On the other hand, the data on a blockchain is immutable once published. It is therefore important to design a blockchain-based platform for government data that allows updates to the information, and tracks them in an  immutable way.
\item {\it Tracking documents.}
%Since the volume of data that the government publishes might be large, it can be difficult to locate a specific document that one has at hand. In particular, the document's hash will probably be ``hidden" in the Merkle tree for the particular block that the document is in, thus making it difficult to find.
Given the relatively large number of documents published by governments and the fact that storing records in the blockchain is expensive, it is not reasonable to store a different identifier for each document in the blockchain. Instead, one could store enough information to allow for irrefutable timestamped proofs of existence for documents using less space in the blockchain (e.g. use a Merkle tree and only store its root in the blockchain). Unfortunately, this introduces the need to have additional information for tracking a document, as the document itself is not enough to produce an identifier that can be verified in the blockchain.
\francisco{I find this Merkle tree occurrence too son.}\martin{I re-wrote this part considering that the issue is not the large amount of data but the fact that the documents' hashes cannot be verified without a certificate}
%Furthermore, it might be difficult to verify if this particular document was ever ``committed" to a public backup of the data described in point one above, since the document's hash will probably be ``hidden" in the Merkle tree for the particular block it is in.

\item {\it Clairvoyance attacks.} Note that using a blockchain-based solution the government can commit to never changing its secret and sensitive data by publishing their hashes. However, it might be possible for a misbehaving government agent to publish various versions of the same document in the blockchain, and in the case of a dispute reveal only the one suiting him best. Since the data itself is published through cryptographic hashes, it will be impossible to differentiate such versions.
%\item {\it Infrastructure issues.} Finally, we have to address the way that government data is created and stored, and the way that particular hash functions can be made vulnerable is not used properly. 
\end{itemize}

In this paper we describe the components of a platform for publishing government data that aims at resolving the technical challenges outlined above. In particular, 
we resolve the credibility issue by publishing occasional hashes of the latest blocks on a publish blockchain supporting smart contracts (in our case Ethereum, but the solution can be extended to many similar platforms). This will also allow us to track documents in a similar way. On the other hand, the persistence issue is solved by designing a specialized protocol for publishing the data that issues a smart contract based certificates for tracking all the updates to a document and to confirm its most recent version. Finally, the problem of clairvoyance attacks is discussed, and we show how clairvoyance attacks be avoided by sacrificing the flexibility of the type of documents that can be stored by the system.

%\francisco{I am a bit skeptic about this phrase. I would rather say that clairvoyance attacks can be avoided, but at the expense of flexibility (you have to impose some format of the target documents, like json objects as in \cite{energiaabierta})}

We would like to note that this paper is certainly not the first one to consider blockhain-based technologies for managing government data. Indeed, this idea has been around for a while, and ranges from fully fledged public blockchains recording every action of the government \cite{eushit,ibmshit}, to smaller scale solutions that track a fixed series of data items \cite{energiaabierta}. What we believe makes our solution technically interesting is the idea of allowing the data to be updated and secured using smart contracts. We also do not provide only a conceptual description of our solution, but also specify the components needed to make it work on a concrete platform (Ethereum in our case). We also discuss some technical issues that have to be handled in this specific application scenario.

In what follows, we start by recapping the basic notions about hash functions, public signatures, and providing a simple protocol for publishing government data using a hash-list. We then move onto providing solutions to each of the technical issues described above.



\section{A basic solution}
\label{sec:basic}
% !TEX root = GovChain.tex

In this section we describe a basic infrastructure for publishing government data that allows anyone with a single data entry to challenge the veracity of the data in case it was manipulated. We also provide a simple protocol to increase the trust of the general public in the data provided by the government. We achieve both of these objectives without modifying the way that the government currently publishes or stores their data. Let us first recall some basic cryptographic primitives we will be using in the rest of this paper.

\medskip
\noindent{\bf Cryptographic primitives.} The first cryptographic primitive we will be using is that of {\em cryptographic (digital) signatures}. Roughly, a digital signature protocol consists of the following three components:
\begin{itemize}
\item A pair $(\sk,\pk)$, where $\sk$ is the secret key known only to a user, and used to generate signatures, and $\pk$ is the public key is given to everyone in order to verify that a message's signature was indeed signed by the owner of the secret key $\sk$.
\item A method $\sign(\sk,\m)$, allowing a holder of $\sk$ to generate a valid signature for $\m$. 
\item A method $\verify(\pk,\m,\mathit{sig})$ for verifying if signature $\mathit{sig}$ was indeed produced with $\sk$ for $\m$.
\end{itemize}
The signature protocol is required to be sound. This means that it verifies correctly, in the sense that $\verify(\pk,m,\sign(\sk,m))$ is always true, and that signatures are unforgeable without the secret key $\sk$. There are many digital signature schemes currently in use, for more details we refer the reader to \cite{KatzLindell2014}. We also point out that a large number of governments have adopted digital signatures as standard, mandatory procedures.

The second primitive we recall is that of {\em cryptographic hash functions}. Roughly speaking, a cryptographic hash function is any function $h$ that takes as input an arbitrary sized document, and produces a fixed size output, can be efficiently computable, and satisfies the following three properties:
\begin{itemize}
\item {\em Hiding.} For any given input $x$, it is infeasible to compute $x$ with knowledge of $h(x)$ only.\footnote{Strictly speaking, this also includes a random number $r$ coming from a distribution with high min-entropy, and being given $h(r||x)$ instead of $h(x)$.}
\item {\em Second preimage resistance.} Given input $x_1$, it is unfeasible to find a different input $x_2$ satisfying $h(x_1)=h(x_2)$.
\item {\em Collision resistance.} It is infeasible to efficiently find two different inputs $x$ and $y$, such that $h(x)=h(y)$.
\end{itemize}
The output of a hash function is called a digest, a hash value or simply a hash. Notice that the hiding property of the hash function allows us to use $h(x)$ as a secure digest of our document $x$, since we can not reconstruct $x$ from $h(x)$ alone, and the value $h(x)$ is of fixed size.
On the other hand, second preimage resistance ensures that to prove to someone who has only $h(x)$ that we have $x$, we only need to provide said $x$. 

There are a number of cryptographic hash functions currently in use. To the date of this writing, most protocols adopt the Secure Hash Algorithms\cite{sha_standard} approved by the National Institute of Standards and Technology (NIST) of the US government, which include SHA-256 and SHA-512 from the SHA-2 family, but other hash functions are considered equally secure and even more efficient. We highlight the BLAKE2 algorithm, released after the SHA-3 competition, which is more secure and faster than SHA-1 and SHA-2. In fact, on most common hardware settings, it is the fastest secure hash function available today, becoming the most popular non-NIST-standard hash algorithm, supported by major cryptographic libraries. For a more detailed treatment of cryptographic hash functions see \cite{sha_standard,aumasson,sha3zoo,bitcoinbook}. %\francisco{Took the liberty of changing this paragraph (from ``To the day...'' on) and adding some references. Feel free to come back to previous versions}

%%%OLD:
%\medskip
%\noindent{\bf Hash lists and Merkle trees.} 
%Let $h$ be a fixed cryptographic hash function. A blockchain is an ordered list of data $[d_1,d_2,\dots, d_k]$ such that every element of the list contains the hash value of the previous element, \ie $h(d_1)\in d_2$. In a blockchain, modifying the hash value of an element affects the hash values of all succesive nodes, and because of the second preimage resistant of $h$, it is unfeasible to replace an element of the list without having to update the list. Therefore, if the last value of the list is permanently monitored, there is virtually no way of modifying an element of the list while keeping the structure of blockchain and not being detected. This is one of the core ideas behind Bitcoin's protocol \cite{whitepaper}, but it has been used widely in a number of precedent applications. \francisco{An actual example from chilean banks follows. Sometimes a PRNG instead of a hash function is used, but it still applies. If you think this example doesn't add to the discussion, just remove it.} For instance, to add an extra layer of security on transactions, some banks provide users small, offline devices that can store one 256-bit value $x$ and show the first six digits of $x$ to the user. Every four or five minutes, the device replaces $x$ with $h(x)$ for some (assumed public) hash function $h$. When a user is about to validate a transaction, the bank asks the six digits visible to the user, which proves possession of the said device with good probability. However, if an attacker knowing $h$ extracts the whole number once (or even only a sufficent amount of bits), they can predict all subsequent values displayed at the device, and successfully pass tests. This is why, in most scenarios, a blockchain needs some source of randomness or interaction with some environment in order to be unpredictable.

\medskip
\noindent{\bf Blockchains and Merkle trees.} 
Let $h$ be a fixed cryptographic hash function. A {\em blockchain} storing data items $d_1,d_2,\dots, d_k$ (in that order) is a sorted list $[B_1,B_2,\ldots ,B_k]$ of elements, also called blocks, satisfying that each $B_i$ contains the document $d_i$, and the value $h(B_{i-1})$. In a blockchain, modifying the hash value of an element affects the hash values of all successive nodes, and because of the second preimage resistance of $h$, it is infeasible to replace an element of the list without having to update all successive elements. Therefore, if the last value of the list is permanently monitored, there is virtually no way of modifying an element of the list while keeping the blockchain structure without being detected. We illustrate this idea in Figure~\ref{hashlist}.


\begin{figure}
\resizebox{\columnwidth}{!}{
\centering
\begin {tikzpicture}[-latex ,auto ,node distance =1 cm and 1.8cm ,on grid, semithick, state/.style ={ rectangle , draw, minimum width =0.9 cm}]
\node[state] (0) {\begin{tabular}{c} $\bot$ \\ $d_1$ \end{tabular}};
\node[state] (1) [right =2.2cm of 0] {\begin{tabular}{c} $h(B_1)$ \\ $d_2$ \end{tabular}};
\node[state] (2) [right =2.5cm of 1] {\begin{tabular}{c} $h(B_2)$ \\ $d_3$ \end{tabular}};
\node[state] (3) [right =2.5cm of 2] {\begin{tabular}{c} $h(B_3)$ \\ $d_4$ \end{tabular}};
\node (etc) [right =of 3] {};

\node (1d) [above =20pt of 0]{$B_1$};
\node (2d) [above =20pt of 1]{$B_2$};
\node (3d) [above =20pt of 2]{$B_3$};
\node (4d) [above =20pt of 3]{$B_4$};

\path (1) edge (0);
\path (2) edge (1);
\path (3) edge (2);
\path (etc) edge [bend left = 0] node[below]{...}(3);

%\path (1d) edge [bend left = 0] (1);
%\path (2d) edge [bend left = 0] (2);
%\path (3d) edge [bend left = 0] (3);

\end{tikzpicture}
}
\caption{A blockchain. We use $\bot$ to denote the default value used as the previous hash in the first block. The arrows in the list are symbolic, as the connection of a block $B_i$ with the block $B_{i-1}$ is actually established through the hash value $h(B_{i-1})$ stored in $B_i$.}
\label{hashlist}
\end{figure}

A {\em Merkle tree} is a data structure that provides a single hash value, called the Merkle root, that is cryptographically linked to the set of documents that \emph{occur} in the tree. This means that for every such document there is an undeniable proof showing that such document was part of the tree when it was created. Moreover, the proof corresponding to each document is logarithmic in size with respect to the total number of documents in the tree.

The properties of Merkle trees imply that committing to a Merkle root is a commit to all documents occurring in the tree, which can be arbitrarily many. Given a set $d_1,\ldots ,d_k$ of documents, a Merkle tree corresponding to these documents, also denoted $MT(d_1,\ldots ,d_k)$, is constructed by grouping the elements in pairs (the final element being duplicated in case that $k$ is odd), computing the hash value of the concatenation of each pair, and applying this recursively to create a complete binary tree whose leaves are labelled with these hash values. Each node whose childern are labelled with $x_1$ and $x_2$ is then labelled with the value $h(x_1 || x_2)$, with $||$ denoting concatenation. We illustrate this construction in Figure~\ref{merkle_fig}. To show that one particular document is in the Merkle tree, we just need to provide the list of siblings of nodes that appear on the path from the root of the tree to this document. For instance, assuming knowledge of the hash value at the root we can prove that the document $d_1$ belongs to the tree from Figure~\ref{merkle_fig} just by providing $d_1$, $x_2$ and $h(x_3||x_4)$. This way we can compute $x_1=h(d_1)$, the value $h(x_1||x_2)$, and the value of the root node.

A final interesting property of Merkle trees is that if we sort the hashes of documents, we can furthermore provide a logarithmic proof to show that a document does not belong to the tree.

\begin{figure}
\resizebox{\columnwidth}{!}{
\centering
\begin {tikzpicture}[-latex ,auto ,node distance =1 cm and 1.8cm ,on grid, semithick, state/.style ={ rectangle , draw, minimum width =0.9 cm}]
% First floor 
\node[state] (d1) {$d_1$};
\node[state] (d2) [right =3cm of d1] {$d_2$};
\node[state] (d3) [right =3cm of d2] {$d_3$};
\node[state] (d4) [right =3cm of d3] {$d_4$};
% Second floor
\node[state] (h1) [above =1.25 of d1] {$x_1=h(d_1)$};
\node[state] (h2) [above =1.25 of d2] {$x_2=h(d_2)$};
\node[state] (h3) [above =1.25 of d3] {$x_3=h(d_3)$};
\node[state] (h4) [above =1.25 of d4] {$x_4=h(d_4)$};
% Third floor
\node[state] (h12) [above right =1.25 and 1.5 of h1] {$h(x_1 || x_2)$};
\node[state] (h34) [above right =1.25 and 1.5  of h3] {$h(x_3 || x_4)$};
% Root
\node[state] (root) [above right =1.25 and 3 of h12] {\small Root};

% First floor paths
\path (h1) edge node[left]{}(d1);
\path (h2) edge node[left]{}(d2);
\path (h3) edge node[left]{}(d3);
\path (h4) edge node[left]{}(d4);

% Second floor paths
\path (h12) edge node{} (h1);
\path (h12) edge node{} (h2);
\path (h34) edge node{} (h3);
\path (h34) edge node{} (h4);

% Third floor paths
\path (root) edge node{} (h12);
\path (root) edge node{} (h34);

\end{tikzpicture}
}
\caption{A Merkle tree with document list $[d_1,d_2,d_3,d_4]$ and hash function $h$. Here, the root contains the value $h(h(x_1||x_2)||h(x_3||x_4))$, which is cryptographically linked to all documents.}% In order to prove that $d_4$ is present on the tree, one needs to check the validity of the blockchain $[x_4,h(x_3||x_4),\mathrm{root}]$ with knowledge of $x_3$ and $h(x_1||x_2)$ only.}
\label{merkle_fig}
\end{figure}

\medskip
\noindent{\bf Basic steps to gain trust.} Now that we understand the basic data structures that guarantee immutability, we can show how they are used to provide compact certificates showing that no data has been manipulated. The basic idea is quite simple: the government will keep on publishing their data as they do at the present, but at regular intervals, they will also publish a compact cryptographic digest of the new data. Assume that $d_1,d_2,\ldots ,d_k$ is the list of the documents published during one working day and that they are published in order. At the end of the day, the government will create a Merkle tree $MT(d_1,\ldots ,d_k)$ as previously described. At this point, the government publishes $MT(d_1,\ldots ,d_k)$, together with the cryptographic signature of this data corresponding to the public key under which the government agreed to publish their data. The fact that the signature matches the data in the Merkle tree and the government's public key, allows anyone to easily verify that the published root is legitimate. Moreover, if someone is only interested in certain documents, they can dismiss all other documents and simply keep the cryptographic proof that links the root of the Merkle tree with his document of interest.

As explained above, anyone who witnesses the above-described process will have an irrefutable proof of the exact set of documents published that day (assuming that the Merkle root cannot be tampered with). If the Merkle root published by the government does not match the one computed using this method, a verifier can know for certain that the government is not accting correctly. Moreover, if the government published the correct Merkle root but later on states any false fact about the data published that day, a verifier will be able to provide an irrefutable proof, exposing the government.

Notice that if we want to be able to detect which particular documents were modified, we would have to store all document's hashes in the same order as they were used for constructing the tree. The size of this will be tens of megabytes for a list of one million documents when using SHA-256 as the hash function, so one might argue that even this is a reasonable amount of storage needed to pinpoint which particular documents had been manipulated. Moreover, even if the verifier does not have any information but the correct Merkle root, the government should be able to provide a succinct proof for the correctness of every document that was indeed part of the Merkle tree.
%One can further restrict the amount of needed storage by storing fewer layers of the Merkle tree (\ie if one trims the original leaves the size will roughly half), but also losing the granularity at which manipulations can be detected. 
%For instance, if we store the tree up to the next to last level, we can check if one of the two adjacent documents in the list $d_1,\ldots ,d_n$ have been modified, but will not be able to say which one (Figure \ref{fig-trim}). 
%For example, if we only store the root and its two children of the Merkle tree from Figure \ref{merkle_fig}, we can detect if there was a modification in the pair $d_1,d_2$, or the pair $d_3,d_4$, but we will not be able to say which of the two documents in the pair was modified.
%Based on the volume of the published data, the government might also wish to publish these partial trees and sign them, in order to make it easier for a person to store this information.
%Martin: esta discusión no me parece muy clara. Si yo guardo el hash de una concatenación no puedo verificar si alguien modificó algo, porque no tengo idea qué había debajo de mi hash.

It is possible to do better by utilizing the same idea that Bitcoin uses to store its blocks \cite{whitepaper}, and encode and/or publish the hash of the previous day's data in the current day's digest. That is, we can create a blockchain of the Merkle roots published each day. This would mean that storing one day's Merkle root (or the entire tree) would not only ensure that the data of that particular day can not be manipulated, but it would provide the same good properties for all data in the past. Since each tree would also contain the hash value of the previous day's digest, one can simply follow this chain day after day, and see if any of the hash values do not match. 
This simple concept is illustrated in Figure \ref{fig-merkle-chain}. Notice that there are two structures that are connected here. On the bottom level, we have the Merkle tree constructed from the day's documents. The top level blockchain is then constructed using the day's Merkle root as its data. That is, each element of this blockchain contains the Merkle root of the current day, plus the hash value of the previous day's element. The only caveat of this structure is that now certificates might become linear in the number of blocks in the blockchain. Although this can be avoided (e.g. by using skip lists) we do not describe the details here.

In this approach the government would publish, apart from the documents themselves, the Merkle root and the last block of the blockchain. In turn, each block will contain the day's Merkle root as its data, plus the hash of the (entire) previous block.

\begin{figure}
\centering
\begin {tikzpicture}[-latex ,auto ,node distance =1 cm and 2.5 cm ,on grid, semithick, state/.style ={ rectangle , draw, minimum width =0.9 cm}, substate/.style ={ rectangle , draw, minimum width =0.2 cm}]

\usetikzlibrary{shapes}
\node (x1) {};
\node (x2) [right = 2.7cm of x1]{};
\node (x3) [right = 2.7cm of x2]{};
\node (etc) [right = 1.8cm of x3]{};

\node[state] (r1) [below =of x1]{\scriptsize$\mathit{MR}_1$};
\draw [draw=black!50] ($(r1.south west) - (0.4,0.1cm)$) rectangle +(1.75cm, +1.2cm);
\node (h1) [above =15pt of r1]{$\bot$};
\node (bl1) [above =30pt of r1]{$B_1$};

\node[state] (r2) [right = 2.7 cm of r1]{\scriptsize$\mathit{MR}_2$};
\draw [draw=black!50] ($(r2.south west) - (0.4,0.1cm)$) rectangle +(1.75cm, +1.2cm);
\node (h2) [above =15pt of r2]{$h(B_1)$};
\node (bl2) [above =30pt of r2]{$B_2$};


\node[state] (r3) [right = 2.7cm of r2]{\scriptsize$\mathit{MR}_3$};
\draw [draw=black!50] ($(r3.south west) - (0.4,0.1cm)$) rectangle +(1.75cm, +1.2cm);
\node (h3) [above =15pt of r3]{$h(B_2)$};
\node (bl3) [above =30pt of r3]{$B_3$};

% First childs
\node[substate] (c11) [below left  = 0.9 of r1]{};
\node[substate] (c12) [below right = 0.9 of r1]{};
\node[substate] (c21) [below left  = 0.9 of r2]{};
\node[substate] (c22) [below right = 0.9 of r2]{};
\node[substate] (c31) [below left  = 0.9 of r3]{};
\node[substate] (c32) [below right = 0.9 of r3]{};

% Second childs
\node[substate] (c111) [below left  = 0.6 and 0.2 of c11]{};
\node[substate] (c112) [below right = 0.6 and 0.2 of c11]{};
\node[substate] (c121) [below left  = 0.6 and 0.2 of c12]{};
\node[substate] (c122) [below right = 0.6 and 0.2 of c12]{};

\node[substate] (c211) [below left  = 0.6 and 0.2 of c21]{};
\node[substate] (c212) [below right = 0.6 and 0.2 of c21]{};
\node[substate] (c221) [below left  = 0.6 and 0.2 of c22]{};
\node[substate] (c222) [below right = 0.6 and 0.2 of c22]{};

\node[substate] (c311) [below left  = 0.6 and 0.2 of c31]{};
\node[substate] (c312) [below right = 0.6 and 0.2 of c31]{};
\node[substate] (c321) [below left  = 0.6 and 0.2 of c32]{};
\node[substate] (c322) [below right = 0.6 and 0.2 of c32]{};

\node (etc111) [below = 0.3 of c111]{$\vdots$};
\node (etc112) [below = 0.3 of c112]{$\vdots$};

\node (etc121) [below = 0.3 of c121]{$\vdots$};
\node (etc122) [below = 0.3 of c122]{$\vdots$};

\node (etc21) [below = 0.9 of c21]{$\vdots$};
\node (etc22) [below = 0.9 of c22]{$\vdots$};
\node (etc31) [below = 0.9 of c31]{$\vdots$};
\node (etc32) [below = 0.9 of c32]{$\vdots$};

% Documents

\node (ghost1) [below left = 1.2 and 0.3 of c11]{};
\node (ghost2) [right = 1 of ghost1]{};
\node (ghost3) [right = 1 of ghost2]{};

\node (d1) [below left  = 0.8 and 0.3 of ghost1]{$d_1$};
\node (d2) [right = 0.5 of d1]{$d_2$};
\node (d3) [below left  = 0.8 and 0.25 of ghost2]{$d_3$};
\node (d4) [right = 0.5 of d3]{$d_4$};
\node (d5) [right = 0.5 of d4]{...};
\node (d6) [right = 0.5 of d5]{$d_k$};

%\path (r1) edge node {} (x1);
%\path (r2) edge node {} (x2);
%\path (r3) edge node {} (x3);
%\path (r3) edge node {} (x3);

%\path (x1) edge node {} (x2);
%\path (x2) edge node {} (x3);
%\path (x3) edge node {} (etc);

\path (r1) edge node {} (c11);
\path (r1) edge node {} (c12);
\path (r2) edge node {} (c21);
\path (r2) edge node {} (c22);
\path (r3) edge node {} (c31);
\path (r3) edge node {} (c32);

\path (c11) edge node {} (c111);
\path (c11) edge node {} (c112);
\path (c12) edge node {} (c121);
\path (c12) edge node {} (c122);

\path (c21) edge node {} (c211);
\path (c21) edge node {} (c212);
\path (c22) edge node {} (c221);
\path (c22) edge node {} (c222);

\path (c31) edge node {} (c311);
\path (c31) edge node {} (c312);
\path (c32) edge node {} (c321);
\path (c32) edge node {} (c322);

\path (ghost1) edge node {} (d1);
\path (ghost1) edge node {} (d2);
\path (ghost2) edge node {} (d3);
\path (ghost2) edge node {} (d4);
\path (ghost3) edge node {} (d6);

\node (ar1) [above right = 10pt and 0.77cm of r1] {};
\node (ar2) [above right = 10pt and 1.95cm of r1] {};
\node (ar3) [above right = 10pt and 3.47cm of r1] {};
\node (ar4) [above right = 10pt and 4.65cm of r1] {};

\path (ar2) edge node {} (ar1);
\path (ar4) edge node {} (ar3);

\end{tikzpicture}
\caption{Each day, a Merkle tree is constructed using that day's published documents $d_1,d_2,\dots, d_k$. The root of this tree is then hashed along with the hash value of the previous day, and the output value is broadcast along with the published documents. The result is a monitorable blockchain $[B_1,B_2,\dots]$.}
\label{fig-merkle-chain}
\end{figure}



\medskip
\noindent{\bf Increasing credibility.} Of course, the solution we just proposed suffers from a similar issue as just publishing government data without the additional blockchain and the Merkle trees. Namely, if no one monitors and keeps track of the blockchain or Merkle trees, the government is free to change the data, recompute the new Merkle trees and the blockchain, and publish them as if they were the original ones. Similarly, if someone does possess a single document they wish to show was changed, even if they have the Merkle tree for that day, it would still be a dispute between the government (claiming that the data that is presently published is the correct one) and the document holder, so it could be difficult to decide whose claim is the actual truth.

A rather simple solution to this problem is to involve into this process non-profit organizations that wish to monitor how transparent is the government in their data publishing practices. More precisely, such organizations would guard some degree of the meta information published by the government, and periodically download the government's documents for some particular day, compute their hash values, and check that the Merkle tree obtained in this way matches the one published by the government. In particular, once the government publicly announces its schema for publishing the data and additional security certificates (Merkle trees and blockchains) it would be simple to build an automated tool that helps organizations verify the correctness of the data-publishing process. Depending on  the organization's capacity, the tool could provide various levels of storage and verification requirements. In particular, the tool would support:
\begin{itemize}
\item Storing only the hash of the last block from the solution proposed in Figure \ref{fig-merkle-chain}. This is a very lightweight solution that can be supported even on mobile devices, as it requires storing only one hash value per day. Using this information, a day's data can be verified for modifications at a later date by downloading and hashing it. We can also detect whether the data on any day previous to this one has been modified.
\item Storing the entire Merkle tree, plus the top level blockchain of Merkle roots. As explained above, apart from allowing to detect whether the data of some particular day has been modified, this also allows us to  tracing which documents suffered changes.
%\item A hybrid solution storing the Merkle tree up to some level, allowing less granularity than storing the entire tree.
%Martin: Tengo la misma preocupación q arriba.
\item In terms of verification, the tool would provide support to verify the data of a particular day by downloading it, hashing it, and check that the resulting Merkle tree is correct. Additionally, the tool could verify backwards compatibility by also following the hashes of previous day's data, and checking that they match.
\end{itemize}

Note that in terms of storage, none of these solutions would be very expensive, and can in a sense serve as guarantors of the published data on government's pages. Additionally, if the amount of these organizations is sufficiently large, and in particular if they are not collaborating, it becomes unfeasible for a malignant agent to subvert all of their data. Furthermore, the fact that the lightest level of this solution could even be stored on smartphones makes it possible for interested individuals to guard the security certificates and check their veracity later, or effectively detect malicious activities.



\section{Technical issues and how to address them}
\label{sec:tech}
% !TEX root = GovChain.tex


While the basic solution we propose goes in the correct direction, it still suffers from some deficiencies which can be addressed (at least partially) on a technological level. In this section we discuss how to further increase the credibility of this way of publishing data, how to deal with updates to the data and checking currently valid versions, as well as how to deal with issues arising when there is a dispute over the \emph{real} history of data. %\francisco{Didn't understand this last point. }

\subsection{Using an established blockchain to achieve credibility.}
% !TEX root = GovChain.tex

The idea of having many different organizations backing up the Merkle trees published by the government works well, however it is still a solution suffering from some weaknesses. For instance, when there are very few such organizations, or individuals willing to guard (a portion of) the security certificates needed to verify the correctness of the data, they could be easily subverted, or give little additional credibility. Overall, one might argue that this solution is easily susceptible to corruption, since the participating organizations do not have any (material) gain from doing their duty correctly. Moreover, a malicious government agent might be able to bribe the participating organizations or just change the data as needed and persecute anyone trying to disclose the truth.
%to change the (Merkle tree and blockchain) data as needed, pretending that it was correct from the start. %Notice that individuals who guarded (a part of) the security certificates, would have even less power in this case, since there is more trust in established organizations, than in individuals.

The solution to this problem is quite simple: publish the top level blockchain from Figure \ref{fig-merkle-chain} announced by the government on a well established public blokchain. That is, in addition to the government and the monitoring agencies publishing this data on their web pages, each day's block is sent to a public blockchain. In case that the blockhcain used is sufficiently secure, as Bitcoin or Ethereum are these days, it is practically unfeasible to change the data once it has been published. In this solution the government, in addition to publishing the actual data, also commits a {\em single} signed transaction to e.g. Ethereum's blockchain, backing up the blockchain of Merkle roots (see Figure \ref{fig-merkle-blockchain}).
% containing the hash value of the Merkle root, hashed together with the hash value of previous day's data.

This amounts to the most lightweight solution proposed in Section \ref{sec:basic}, where only the root is monitored. While this does not allow to trace modifications of single documents, it does allow to verify proofs for valid documents. In other words, one can generate a certificate to convince anyone that a document was indeed published on a particular day. The restriction that a single transaction is published per day is there to ensure that the government commits to a particular view of the data for that day. This way, it is straightforward to locate the entire history on the blockchain by looking at the information stored in the corresponding smart contract, and guarding the hash values these transactions commit to. This solution is extremely lightweight for the Ethereum network (which makes it very cheap) while allowing to audit a government publishing an arbitrarily large number of documents per day.

\begin{figure}
%\label{fig-merkle-blockchain}
\centering
\begin {tikzpicture}[-latex ,auto ,node distance =1 cm and 2.5 cm ,on grid, semithick, state/.style ={ rectangle , draw, minimum width =0.9 cm}, substate/.style ={ rectangle , draw, minimum width =0.2 cm}]

\usetikzlibrary{shapes}
\node (x1) {};
\node (x2) [right = 2.7cm of x1]{};
\node (x3) [right = 2.7cm of x2]{};
\node (etc) [right = 1.8cm of x3]{};

\node[state] (r1) [below =of x1]{\scriptsize$\mathit{MR}_1$};
\draw [draw=black!50] ($(r1.south west) - (0.4,0.1cm)$) rectangle +(1.75cm, +1.2cm);
\node (h1) [above =15pt of r1]{$\bot$};
\node (bl1) [above =30pt of r1]{$B_1$};

\node[state] (r2) [right = 2.7 cm of r1]{\scriptsize$\mathit{MR}_2$};
\draw [draw=black!50] ($(r2.south west) - (0.4,0.1cm)$) rectangle +(1.75cm, +1.2cm);
\node (h2) [above =15pt of r2]{$h(B_1)$};
\node (bl2) [above =30pt of r2]{$B_2$};

\node[style={ellipse}] (BC) [above =2.5cm  of r3] {Ethereum};

\path (bl1) edge[dotted, bend left=25] node {} (BC);
\path (bl2) edge[dotted, bend left=25] node {} (BC);
\path (bl3) edge[dotted, bend right=25] node {} (BC);

\node[state] (r3) [right = 2.7cm of r2]{\scriptsize$\mathit{MR}_3$};
\draw [draw=black!50] ($(r3.south west) - (0.4,0.1cm)$) rectangle +(1.75cm, +1.2cm);
\node (h3) [above =15pt of r3]{$h(B_2)$};
\node (bl3) [above =30pt of r3]{$B_3$};

% First childs

\node[substate] (c11) [below left  = 0.9 of r1]{};
\node[substate] (c12) [below right = 0.9 of r1]{};
\node[substate] (c21) [below left  = 0.9 of r2]{};
\node[substate] (c22) [below right = 0.9 of r2]{};
\node[substate] (c31) [below left  = 0.9 of r3]{};
\node[substate] (c32) [below right = 0.9 of r3]{};

% Second childs

\node[substate] (c111) [below left  = 0.6 and 0.2 of c11]{};
\node[substate] (c112) [below right = 0.6 and 0.2 of c11]{};
\node[substate] (c121) [below left  = 0.6 and 0.2 of c12]{};
\node[substate] (c122) [below right = 0.6 and 0.2 of c12]{};

\node[substate] (c211) [below left  = 0.6 and 0.2 of c21]{};
\node[substate] (c212) [below right = 0.6 and 0.2 of c21]{};
\node[substate] (c221) [below left  = 0.6 and 0.2 of c22]{};
\node[substate] (c222) [below right = 0.6 and 0.2 of c22]{};

\node[substate] (c311) [below left  = 0.6 and 0.2 of c31]{};
\node[substate] (c312) [below right = 0.6 and 0.2 of c31]{};
\node[substate] (c321) [below left  = 0.6 and 0.2 of c32]{};
\node[substate] (c322) [below right = 0.6 and 0.2 of c32]{};

\node (etc111) [below = 0.3 of c111]{$\vdots$};
\node (etc112) [below = 0.3 of c112]{$\vdots$};

\node (etc121) [below = 0.3 of c121]{$\vdots$};
\node (etc122) [below = 0.3 of c122]{$\vdots$};

\node (etc21) [below = 0.9 of c21]{$\vdots$};
\node (etc22) [below = 0.9 of c22]{$\vdots$};
\node (etc31) [below = 0.9 of c31]{$\vdots$};
\node (etc32) [below = 0.9 of c32]{$\vdots$};

% Documents

\node (ghost1) [below left = 1.2 and 0.3 of c11]{};
\node (ghost2) [right = 1 of ghost1]{};
\node (ghost3) [right = 1 of ghost2]{};

\node (d1) [below left  = 0.8 and 0.3 of ghost1]{$d_1$};
\node (d2) [right = 0.5 of d1]{$d_2$};
\node (d3) [below left  = 0.8 and 0.25 of ghost2]{$d_3$};
\node (d4) [right = 0.5 of d3]{$d_4$};
\node (d5) [right = 0.5 of d4]{...};
\node (d6) [right = 0.5 of d5]{$d_k$};

%\path (r1) edge node {} (x1);
%\path (r2) edge node {} (x2);
%\path (r3) edge node {} (x3);
%\path (r3) edge node {} (x3);

%\path (x1) edge node {} (x2);
%\path (x2) edge node {} (x3);
%\path (x3) edge node {} (etc);

\path (r1) edge node {} (c11);
\path (r1) edge node {} (c12);
\path (r2) edge node {} (c21);
\path (r2) edge node {} (c22);
\path (r3) edge node {} (c31);
\path (r3) edge node {} (c32);

\path (c11) edge node {} (c111);
\path (c11) edge node {} (c112);
\path (c12) edge node {} (c121);
\path (c12) edge node {} (c122);

\path (c21) edge node {} (c211);
\path (c21) edge node {} (c212);
\path (c22) edge node {} (c221);
\path (c22) edge node {} (c222);

\path (c31) edge node {} (c311);
\path (c31) edge node {} (c312);
\path (c32) edge node {} (c321);
\path (c32) edge node {} (c322);

\path (ghost1) edge node {} (d1);
\path (ghost1) edge node {} (d2);
\path (ghost2) edge node {} (d3);
\path (ghost2) edge node {} (d4);
\path (ghost3) edge node {} (d6);

\node (ar1) [above right = 10pt and 0.77cm of r1] {};
\node (ar2) [above right = 10pt and 1.95cm of r1] {};
\node (ar3) [above right = 10pt and 3.47cm of r1] {};
\node (ar4) [above right = 10pt and 4.65cm of r1] {};

\path (ar2) edge node {} (ar1);
\path (ar4) edge node {} (ar3);

\end{tikzpicture}
\caption{In addition to the broadcast of single documents and the Merkle root, the blockchain $[B_1,B_2,\dots]$ is published element-wise to a well established blockchain. Anyone having access to a single path in a tree and the previous day hash can verify the integrity of the whole tree.}
\label{fig-merkle-blockchain}
\end{figure}

\lstset{basicstyle=\footnotesize}
\lstset{
  numbers=left,
  stepnumber=1,    
  firstnumber=1,
  numberfirstline=true
}
\begin{figure*}
\begin{lstlisting}[language=java]
pragma solidity ^0.4.24;

contract GovChain{

    mapping (bytes32 => bytes32) public roots;
    mapping (bytes32 => bytes32) public blocks_chain;
    mapping (bytes32 => uint) public timestamps;
    bytes32 public last_block_hash;
    address public publisher_address;

    function GovChain() public {
        publisher_address = msg.sender;
        last_block_hash = 1;
    }

    function add_root(bytes32 merkle_root) public {
        // Check that the sender is correct
        assert(msg.sender == publisher_address);
        // Check the elapsed time is at least 20 hours
        assert(block.timestamp - timestamps[last_block_hash] > 72000);        
        // Hash the concatenation of the previous block's hash and the new Merkle root
        bytes32 new_block_hash = sha256(last_block_hash, merkle_root);
        // Check that this hash has not been uploaded before.
        assert(roots[new_block_hash] == 0);
        roots[new_block_hash] = merkle_root;
        blocks_chain[new_block_hash] = last_block_hash;
        timestamps[new_block_hash] = block.timestamp;
        last_block_hash = new_block_hash;
    }
}

\end{lstlisting}
\caption{Solidity code for backing up the top level blocks of Figure \ref{fig-merkle-chain} on Ethereum's blockchain. Note that the code itself constructs the blockchain, thus giving additional credibility to the blockchain published by the government.}
\label{contract}
\end{figure*}

To make things concrete, we next illustrate how this can be achieved in Ethereum. In Figure \ref{contract} we give a Solidity contract for publishing the blockchain of Merkle roots in a secure way. Solidity \cite{Solidity} is a high level language for describing smart contracts that is then compiled to be executed on Ethereum's virtual machine. The contract we present is to be called by the government publishing the data, and they will be the only ones able to execute the contract (assuming their private key does not get compromized). This is assured by guarding the contract publisher's address (in form of their public key) when the contract is deployed (line 12 of Figure~\ref{contract}), and then checking that it is this particular address calling the contract to upload new values (line 18). The contract itself guards the data corresponding to the block of each day, plus the time they were published. For this we maintain a mapping (i.e. a dictionary) \texttt{roots} linking the hash of each block to the corresponding Merkle root, a mapping \texttt{blocks\_chain} linking the hash of each block to the hash of the previous block of the blockchain (e.g. $h(B_2)$ in $B_3$ of Figure \ref{fig-merkle-chain}), as well as a mapping \texttt{timestamps} that tells us the time each block was published. We also store the hash value of the last block in the variable \texttt{last\_block\_hash}.

The only function that the contract can execute once deployed is for extending the list by a new Merkle root. This is achieved by the function \texttt{add\_root}, which takes the Merkle root value as the input and adds it to the blockchain. Note that our contract allows the blocks to be submitted at most once a day (in line 20 we require the new block to have a delay of at least 20 hours to allow some flexibility on when the day's data is published). If these verifications complete successfully, we then compute the hash value of the new element of our list (line 22), check it was not used before (line 24), and define its root value as the hash value received as input to the function (line 25). We then also set the value of the previous element of the list in line 26, the timestamp in line 27, and update the value of the ultimate element of the list as this one in line 28.

Note that this contract can also serve as the verification that the blockchain published by the government is indeed the correct one, as it recomputes its hash values (line 22 of the contract). This way, we have additional assurance that the government's data about the blockchain is correct, since it is recomputed on Ethereum's blockchain, in addition to being published elsewhere.

Overall, publishing the data on Ethereum's blockchain using a contract presented in Figure~\ref{contract} is a very efficient and lightweight solution. In particular, since submitting Merkle roots is limited to one daily, the cost of this contract will be very low (way below one dollar per day at the current Ethereum/dollar exchange rate), and does not heavily contribute to the congestion of Ethereum's network. Furthermore, having the value of any element of the published blockchain allows us to access it for free by connecting to Ethereum nodes (or running one), and can provide us with the value of that day's Merkle root, date, and the hash value of the previous element of the list. This in turn allows us to obtain data directly from the government and, plus a succinct certificate that should also be provided by the government, verify that we are obtaining the correct information.

%Finally, notice that the idea of monitoring organizations presented in Section \ref{sec:basic} can now be further simplified, as they do not need to keep the entire top level blockchain of Merkle roots any more, but can just occasionally download the hash value of the latest block on this list. Since the list itself is being guarded in the contract on the Ethereum's blockchain, looking up this value allows us to both check that it corresponds to the correct day's data (through the value of the mapping \texttt{timestamps}), that it matches the corresponding Merkle root, and includes the correct hash of the previous day's data. In essence, by committing to this smart contract, the government can not go back and change it's day digest in any way, and there can be no dispute on whether the Merkle root of the day's data provided by the government and the one in possession of a monitoring agency/person is different, modulo the security of Ethereum's blockchain.
%\martin{I highly disagree here. The use of monitoring organizations is now useless for storing what is already stored in the smart contract. However, it is still (very) relevant for on-line verification that the government is doing everything correctly and for storing all of the data in case the government refutes to provide a certificate. Also, these organizations should provide anti-certificates in case the government says something incorrect.}

Finally, notice that the idea of monitoring organizations presented in Section \ref{sec:basic} can now be further strengthened. First, by committing to this smart contract, the government can not go back and change it's day digest in any way, because if it does this, the monitoring agency can easily pinpoint the change based on the data available in the smart contract and the one they store locally. Second, if the government refuses to provide a certificate for a specific document that was originally present, a monitoring agency that also store the Merkle trees for each day (or has a path to the root of the tree where this document was stored) can easily call them out for this and show that the document was indeed present in the government's data for that day, thus making it impossible for documents to disappear.


%\domagoj{This can be further implemented using a smart contract ME IMAGINO QUE SI??? ESCRIBAN DETALLES DE STEALA HASH AQUI.}
%
%\domagoj{IF MONITORING ORGANIZATIONS ARE WELL ESTABLISHED MAYBE MOVE TO A N OUT OF M SIGNATURE TO COMMIT/OR AN ADDITIONAL TRANSACTION THEY PUBLISH WITH N OUT OF M SIGNATURES TO VERIFY THE DAY'S DATA?}\francisco{Or they could actually sign the same contract as the government}
%
%In fact, the only blockchain-related technological need for this is a tool that lets an authorized party to daily publish one short string in a respected blockchain (typically, the signed output of a SHA-256 or BLAKE2 computation into Ethereum or Bitcoin blockchains). This is simple and of negligeable cost. We have witnessed some initiatives aiming to do this since the early years of Bitcoin, but we have yet to see one that is appropriate for government use. Some of them are vulnerable to the well-known length-extension attacks, most of them do not support cryptographic signatures on published hashes, and almost all of them charge the user a large, arbitrary amount for this relatively simple service. 
%
%\francisco{I think this section should include Martin's guarants or guards. Or maybe a separate section on it would be better.}



\subsection{Allowing the data to be updated.}
\label{sec:updates}
% !TEX root = GovChain.tex

It seems counterintuitive at first to use an established, immutable blockchain in order to commit to data that is subject to change often. It is yet a principal requirement of governments for data to be updated, given the complex architecture of hierarchies and the nature of the information the documents may contain. Documents collecting signatures may change daily, spreadsheets may be corrected for errors, and agendas of public projects may be filled with more details as the time passes. The basic solution given above with a small tweak actually allows to perform this.

The solution is based on the following simple idea: there are two types of leafs in our Merkle trees, new documents and new versions of previous documents. Whenever a new version of a previous document is published, it must include the hash value of the \emph{last version} of that same document, thus creating a blockchain structure for the history of each document. Now, since this is not occurring in a smart contract, the government could simply publish two different updates of the same document, thus \emph{branching} a document's history. However, in such a case an interested organization verifying the governments actions would have a trivial proof (verifyiable by anyone) showing that the government is doing something incorrect.

More formally, leafs of a Merkle tree would now look like $h_1,\ldots,h_j,(h_{j+1}:h_{j+1}'), \ldots,(h_n:h_n')$ where $h_1,\ldots,h_j$ are hashes of new documents and $(h_{j+1}:h_{j+1}'), \ldots,(h_n:h_n')$ represent that the document with hash $h_{i}$ is an updated version of the document with hash $h_{i}'$. Note again that there is not enough information in this data structure to verify that the government is behaving correctly. For example, we cannot immediately deduce that $h_{j+1}'$ is indeed the last version of a document (or if was ever published). However, an interested party verifying the publication of documents by the government would easily realize incorrect actions, and moreover would have a succinct and irrefutable proof of the government's errors.




%\subsection{Impossible to erase documents.}
%\input{erasing}


\subsection{Sensitive data and clairvoyance attacks.}
% !TEX root = GovChain.tex


Using the hiding property of a hash function, we might argue that the government might also guarantee for the immutability of their secret data that they do not want to publish. More precisely, assume that the government has a list $d_1,d_2,\ldots ,d_k$ of documents they wish to keep secret, but want to assure the general public that the documents themselves will not be changed not manipulated. For this, the government could simply create the Merkle tree $MT(d_1,\ldots ,d_k)$ that contains the hash values $h(d_1),\ldots ,h(d_k)$ as its leaves, and publish this tree as the safety certificate. If at a later date there is a dispute regarding some issue pertaining to document $d_i$ (e.g. in case that $d_i$ is a contract under question), a judicial body might require the government agency to make this particular document public, and show its membership in the Merkle tree $MT(d_1,\ldots ,d_k)$. In case that the document was manipulated in the meantime, this can now be detected, as the leave hash will not match the path to the root of the tree. Note that this still allows the documents different from $d_i$ to remain secret.

One problem with this solution is that since the documents $d_1,\ldots ,d_k$ are not made public, a malicious government agent might publish two versions of the document: one that suits him and does not reflect signs of foul play, and another that holds the actual true. For example, suppose that the government does not disclose tax certificates of particular entities, but includes these in the Merkle tree. Assume also that a certificate for an entity $X$ was issued stating that the taxes are not in order, and was included in a Merkle tree. An authorized government agent is then bribed to also issue a certificate that the taxes are in order, together with the proof of the inclusion in a Merkle tree (note that this can even be the same tree containing the first certificate). If $X$ is at a later date asked to prove the status of their taxes, they would just provide the certificate showing the taxes to be in order, together with the proof of the inclusion in the corresponding Merkle tree. Since we have no way to detect that the other certificate is also in the tree (due to the data being secret), we can not detect foul play in this case. A similar sort of embezzlement can be executed with contracts stating how much public funds were used to execute a particular activity, or spent on a particular service.

This can be addressed at the Merkle tree construction level, demanding some behaviour from the government and adding a simple verifying procedure to the guarding entities, as follows. Allow the government to input hashes of undisclosed documents in the Merkle tree leafs, but for each secret document, the government must input a public document containing (a) the hash value of the corresponding secret document and (b) some reasonable description thereof. Let us refer to these documents as \textit{affidavits}. In simple terms, an affidavit of the example above must read \textit{``A secret document with hash value $\langle \mbox{hash}\rangle$ has been inserted into this tree. Description: Tax situation of entity X. $\langle\mbox{timestamp}\rangle$, $\langle\mbox{signature}\rangle$''}. 

In order to avoid any ambiguities, we suggest that each affidavit corresponds to exactly one secret document in the same Merkle tree, and that they should be treated as regular public documents. The guarding entities facing such a tree must perform an additional verification: For each hash whose document they cannot access, there must be a affidavit in the same tree containing that hash value, and a description. There is a delicate degree of liberty on what a reasonable description should look like. We suggest that such a description contains the maximum amount of public information, in order to be used later as a proof. For instance, it should allow to trace two contradictory documents (a guarding entity may allow users to search affidavits by description). Should the government give unrelated or incomplete descriptions, this would become evident when a court orders to reveal the underlying document. 

Adding this to previous considerations, Merkle trees could also include elements of the form $(z:a)$ where $z$ is the hash of an undisclosed document (indicated in the body of the publicly available affidavit), and $a$ is the hash value of the affidavit itself.
%
%$$
%\begin{array}{ll}
%\mbox{\it New: } &\{x_1,x_2,\dots,x_n\},\\
%\mbox{\it Updates: }& \{(y_1:y_1'), (y_2:y_2'),\dots, (y_m:y_m')\},\\
%\mbox{\it Undisclosed: }& \{z_1,z_2,\dots,z_k\}\\
%\mbox{\it Affidavits: }& \{a_1,a_2,\dots,a_k\}\\
%\end{array}
%$$
%

%Upon publication of a Merkle root by the government, fully responsible verifiers should download all corresponding data, compute the hash value of each new document, check that each affidavit contains a hash value, and that updates link to a non-updated version of a document. After doing this, they assemble a Merkle tree with all computed data: (i) hashes of new documents, (ii) hashes of updated documents (possibly linking to the hash of the previous version, as discussed in \ref{sec:updates}) and (iii) hash values stated in affidavits. Note that affidavits are treated as regular disclosable documents, whose hash value is therefore included in (i). To finish, this entity verifies that the Merkle root of this tree matches the last published Merkle root in the established blockchain. There are some relatively straightforward details in these verifications which we omit here. 


%\domagoj{add some potential solutions}




\subsection{Tracking, formatting and hash functions.}
% !TEX root = GovChain.tex
	
Finally, we would like to discuss some minor issues that need to be addressed when deploying a platform for publishing government's data in an  open and transparent manner.

First, the solution we propose might make it somewhat difficult to track a particular document. That is, if one simply has a document at hand, and no additional meta data, verifying where this document comes from, in which Merkle tree can we find its hash, or whether this is the latest version of the document might be difficult. To make this simpler, the most elegant solution would be that the API for downloading particular documents published by the government also includes the meta information needed to find the security certificates for a particular document. Furthermore, to track the latest version of the document, we can now turn either to the government's API or to the monitoring agencies, as described in Section \ref{sec:updates}.

%That is, when downloading each document, one would also obtain the root of its Merkle tree, together with the witnessing path of the inclusion in this Merkle tree, plus the tree's address in the Ethereum blockchain. With this information we can now easily check that the document we have was included in the official data published by the government, plus find a proof of this on Ethereum's blokchain. Furthermore, to track the latest version of the document, we can now turn either to the government's API or to the monitoring agencies, as described in Section \ref{sec:updates}.

%Another issue we face when dealing with government records is that they are usually not stored in standardized formats, and most likely use spreadsheet software such as Microsoft Excel, or text processors such as Microsoft Word. One issue with these tools is that they will change the document's hash value every time they open a document, even when no visible changes have been made (i.e. no new character has been added/removed), mostly due to the meta data they store along with the document. A similar issue can occur based on which operating system is used when processing the document. In both cases, a person who downloaded the document might compute its hash, and obtain a value that is different that that when a hash of the downloaded document was computed. To resolve this, one needs to have a clearly specified format in which the government's documents will be published, and inform the people using it of potential issues. Some solutions here include publishing data in binary format, or as pdf.

Another issue we face when dealing with government records is that they are usually not stored in standardized formats, and most likely use spreadsheet software such as Microsoft Excel, or text processors such as Microsoft Word. One issue with these tools is that they will change the document's hash value every time they open a document, even when no visible changes have been made (i.e. no new character has been added/removed), due to the meta data they store. In this case, a person who downloaded the document might compute its hash, and obtain a value that is different than the hash of the downloaded document. To resolve this, one needs to have a clearly specified format in which the government's documents will be published, and inform the people using it of potential issues. Some solutions here include publishing data in binary format, or as pdf.

Next, we also have to be careful with the specifics of the hash function being used. For instance, it is well known that popular choices such as SHA-256 is susceptible to length extension attacks \cite{lengthextension}, thus requiring to use it in a specific way (e.g. by using $h(x||h(x))$ instead of only $h(x)$). Alternatively, one can opt for more secure hash options that have been proved secure against such attacks \cite{keccak}.

Overall, these illustrate just some of the specific issues we will face when designing a platform for transparent publishing of government's data.

%	\francisco{Microsoft attacks, One millisecond attacks, Length extension attacks}

%	\francisco{Also the programmed updates attack (when documents are predictable and hash lists do not include outer data). Is it harmful?}





\section{Potential impact and future directions}
\label{sec:future}
% !TEX root = GovChain.tex

In this paper we describe the main components needed to create a blockchain-based platform for publishing government data and assure it against manipulation. The platform is designed in a non intrusive way, and, assuming that the government keeps copies of its records, requires only publishing a few inexpensive meta data documents that can assure the published data from being manipulated. The publication process can be summarized in the following steps:
\begin{itemize}
\item A list of documents $d_1,\ldots ,d_k$ is published by the government each day and made available to the general public. Some of these documents are new, some are updates to previous versions.
\item In addition to these documents, the government also publishes a Merkle tree with the hashes of this day's documents. This Merkle tree can also contain hashes of non-disclosed documents, in which case one \textit{affidavit} per unrevealed document must be published and incorporated to the tree.
\item Furthermore, the government maintains hash list of the roots of these Merkle trees (Figure \ref{fig-merkle-blockchain}).
\item Finally, the hash list from the previous step is also backed-up on Ethereum's blockchain by sending a single daily transaction to the smart contract of Figure \ref{contract}, updating the hash list with the root of the current tree.
\end{itemize}

Additionally, the government also provides a Web API for retrieving the documents, that allows to download them with the pertinent meta data (e.g. the certificate of their membership in the day's Merkle tree, and how to find their certificate on Ethereum's blockchain). The API might also allow searching for certificates of particular documents.

Finally, monitoring agencies, or interested individuals, have an access to a tool that can further increase the credibility of the security certificates by storing the Merkle trees, or elements of the hash list, and occasionally verifying the correctness of the published data.

The benefits of our approach is that it is non intrusive, since it does not require to change the way that the data is published, but only acts as an add-on via issuing certificates assuring that the data will not be manipulated. Furthermore, it removes the trust issue from a single agent (the government), increases security of government data (by making breaches or errors very easy to detect), and it also standardizes the way people access government's data. Because of these benefits, we believe that building such a blockchain-based platform for publishing governmental data has several advantages to the traditional way of publishing open data, and that deploying this platform can change the trust of the general public in the democratic process by making it much more transparent.

\francisco{Why is not bibliography alphabetical?}
\bibliographystyle{IEEEtran}
\bibliography{refs}

\end{document}
