% !TEX root = GovChain.tex

In this paper we described the main components needed to create a blockchain-based platform for publishing government data and protect it against manipulation. The platform is designed in a non intrusive way, and, assuming that the government keeps copies of its records, requires only publishing some metadata that assures the published data is not being modified. The publication process can be summarized in the following steps:
\begin{itemize}
\item A list of documents $d_1,\ldots ,d_k$ is published by the government each day and made available to the general public. Some of these documents are new, some are updates to previous versions.
\item In addition to these documents, the government also publishes a Merkle tree with the hashes of the day's documents. This Merkle tree can also contain hashes of undisclosed documents, in which case one \textit{affidavit} per document must be published and incorporated to the tree (Section \ref{sec:sensitive}).
\item Furthermore, the government maintains hash list of the roots of these Merkle trees (Figure \ref{fig-merkle-blockchain}).
\item Finally, the hash list from the previous step is also backed-up on Ethereum's blockchain by sending a single daily transaction to the smart contract, updating the hash list with the root of the current tree (Figure \ref{contract}).
\end{itemize}

Additionally, the government also provides a Web API for retrieving the documents, that allows to download them with the pertinent metadata (e.g. the certificate of their membership in the day's Merkle tree, and how to find their certificate on Ethereum's blockchain). This API might also allow searching for certificates of particular documents.

Finally, monitoring agencies, or interested individuals, have an access to a tool that can further increase the credibility of the security certificates by storing the Merkle trees, or elements of the hash list, and frequently verifying the correctness of the published data.

The benefits of our approach is that it is non intrusive, since it does not require to change the way that the data is published, but only acts as an add-on via issuing certificates assuring that the data will not be manipulated. Furthermore, it removes the trust issue from a single agent (the government), increases security of government data (by making breaches or errors very easy to detect), and it also standardizes the way people access government's data. Because of these benefits, we believe that building such a blockchain-based platform for publishing governmental data has several advantages over traditional ways of broadcasting data, and that deploying such platforms can address trust issues of the general public in democratic processes, by making them much more transparent.