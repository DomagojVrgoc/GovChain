% !TEX root = GovChain.tex

In this section we describe a basic infrastructure for publishing government data that allows anyone with a single data entry to challenge the veracity of the data in case it was manipulated. We also provide a simple protocol to increase the trust of the general public in the data provided by the government. We achieve both of these objective without modifying the way that the government currently publishes or stores their data. Let us first recall some basic cryptographic primitives we will be using in the rest of this paper.

\medskip
\noindent{\bf Cryptographic primitives.} The first cryptographic primitive we will be using is that of {\em cryptographic (digital) signatures}. Roughly, a digital signature protocol consists of the following three components:
\begin{itemize}
\item A pair $(\sk,\pk)$, where $\sk$ is the secret key known only to a user, and used to generate signatures, and $\pk$ is the public key is given to everyone in order to verify that a message's signature was indeed signed by the owner of the secret key $\sk$.
\item A method $\sign(\sk,\m)$, allowing a holder of $\sk$ to generate a valid signature for $\m$. 
\item A method $\verify(\pk,\m,\mathit{sig})$ for verifying if signature $\mathit{sig}$ was indeed produced with $\sk$ for $\m$.
\end{itemize}
The signature protocol is required to be sound. This means that it verifies correctly, in the sense that $\verify(\pk,m,\sign(\sk,m))$ is always true, and that signatures are unforgeable without the secret key $\sk$. There are many digital signature schemes currently in use, for more details we refer the reader to \cite{KatzLindell2014}. We also point out that a large number of governments have adopted digital signatures as standard, mandatory procedures.

The second primitive we recall is that of {\em cryptographic hash functions}. Roughly speaking, a cryptographic hash function is any function $h$ that takes as input an arbitrary sized document, and produces a fixed size output, can be efficiently computable, and satisfies the following three properties:
\begin{itemize}
\item {\em Hiding.} For any given input $x$, it is infeasible to compute $x$ with knowledge of $h(x)$ only.\footnote{Strictly speaking, this also includes a random number $r$ coming from a distribution with high min-entropy, and being given $h(r||x)$ instead of $h(x)$.}
\item {\em Second preimage resistance.} Given input $x_1$, it is unfeasible to find a different input $x_2$ satisfying $h(x_1)=h(x_2)$.
\item {\em Collision resistance.} It is infeasible to efficiently find two different inputs $x$ and $y$, such that $h(x)=h(y)$.
\end{itemize}
The output of a hash function is called a digest, a hash value or simply a hash. Notice that the hiding property of the hash function allows us to use $h(x)$ as a secure digest of our document $x$, since we can not reconstruct $x$ from $h(x)$ alone, and the value $h(x)$ is of fixed size.
%\francisco{This is true but lacks some important properties of a secure encryption, maybe change the word encryption here?}. 
On the other hand, second preimage resistance ensures that to prove to someone who has only $h(x)$ that we have $x$, we only need to provide said $x$. 

There are a number of cryptographic hash functions currently in use. To the date of this writing, most protocols adopt the Secure Hash Algorithms\cite{sha_standard} approved by the National Institute of Standards and Technology (NIST) of the US government, which include SHA-256 and SHA-512 from the SHA-2 family, but other hash functions are considered equally secure and even more efficient. We highlight the BLAKE2 algorithm, released after the SHA-3 competition, which is more secure and faster than SHA-1 and SHA-2. In fact, on most common hardware settings, it is the fastest secure hash function available today, becoming the most popular non-NIST-standard hash algorithm, supported by major cryptographic libraries. For a more detailed treatment of cryptographic hash functions see \cite{sha_standard,aumasson,sha3zoo,bitcoinbook}. %\francisco{Took the liberty of changing this paragraph (from ``To the day...'' on) and adding some references. Feel free to come back to previous versions}

%%%OLD:
%\medskip
%\noindent{\bf Hash lists and Merkle trees.} 
%Let $h$ be a fixed cryptographic hash function. A blockchain is an ordered list of data $[d_1,d_2,\dots, d_k]$ such that every element of the list contains the hash value of the previous element, \ie $h(d_1)\in d_2$. In a blockchain, modifying the hash value of an element affects the hash values of all succesive nodes, and because of the second preimage resistant of $h$, it is unfeasible to replace an element of the list without having to update the list. Therefore, if the last value of the list is permanently monitored, there is virtually no way of modifying an element of the list while keeping the structure of blockchain and not being detected. This is one of the core ideas behind Bitcoin's protocol \cite{whitepaper}, but it has been used widely in a number of precedent applications. \francisco{An actual example from chilean banks follows. Sometimes a PRNG instead of a hash function is used, but it still applies. If you think this example doesn't add to the discussion, just remove it.} For instance, to add an extra layer of security on transactions, some banks provide users small, offline devices that can store one 256-bit value $x$ and show the first six digits of $x$ to the user. Every four or five minutes, the device replaces $x$ with $h(x)$ for some (assumed public) hash function $h$. When a user is about to validate a transaction, the bank asks the six digits visible to the user, which proves possession of the said device with good probability. However, if an attacker knowing $h$ extracts the whole number once (or even only a sufficent amount of bits), they can predict all subsequent values displayed at the device, and successfully pass tests. This is why, in most scenarios, a blockchain needs some source of randomness or interaction with some environment in order to be unpredictable.

\medskip
\noindent{\bf Blockchains and Merkle trees.} 
Let $h$ be a fixed cryptographic hash function. A {\em blockchain} storing data items $d_1,d_2,\dots, d_k$ (in that order) is a sorted list $[B_1,B_2,\ldots ,B_k]$ of elements, also called blocks, satisfying that each $B_i$ contains the document $d_i$, and the value $h(B_{i-1})$. In a blockchain, modifying the hash value of an element affects the hash values of all successive nodes, and because of the second preimage resistance of $h$, it is infeasible to replace an element of the list without having to update all successive elements. Therefore, if the last value of the list is permanently monitored, there is virtually no way of modifying an element of the list while keeping the blockchain structure without being detected. We illustrate this idea in Figure~\ref{hashlist}.


\begin{figure}
\resizebox{\columnwidth}{!}{
\centering
\begin {tikzpicture}[-latex ,auto ,node distance =1 cm and 1.8cm ,on grid, semithick, state/.style ={ rectangle , draw, minimum width =0.9 cm}]
\node[state] (0) {\begin{tabular}{c} $\bot$ \\ $d_1$ \end{tabular}};
\node[state] (1) [right =2.2cm of 0] {\begin{tabular}{c} $h(B_1)$ \\ $d_2$ \end{tabular}};
\node[state] (2) [right =2.5cm of 1] {\begin{tabular}{c} $h(B_2)$ \\ $d_3$ \end{tabular}};
\node[state] (3) [right =2.5cm of 2] {\begin{tabular}{c} $h(B_3)$ \\ $d_4$ \end{tabular}};
\node (etc) [right =of 3] {};

\node (1d) [above =20pt of 0]{$B_1$};
\node (2d) [above =20pt of 1]{$B_2$};
\node (3d) [above =20pt of 2]{$B_3$};
\node (4d) [above =20pt of 3]{$B_4$};

\path (1) edge (0);
\path (2) edge (1);
\path (3) edge (2);
\path (etc) edge [bend left = 0] node[below]{...}(3);

%\path (1d) edge [bend left = 0] (1);
%\path (2d) edge [bend left = 0] (2);
%\path (3d) edge [bend left = 0] (3);

\end{tikzpicture}
}
\caption{A blockchain. We use $\bot$ to denote the default value used as the previous hash in the first block. The arrows in the list are symbolic, as the connection of a block $B_i$ with the block $B_{i-1}$ is actually established through the hash value $h(B_{i-1})$ stored in $B_i$.}
\label{hashlist}
\end{figure}

A {\em Merkle tree} is a data structure that provides a single hash value, called the Merkle root, that is cryptographically linked to the set of documents that \emph{occur} in the tree. This means that for every such document there is an undeniable proof showing that such document was part of the tree when it was created. Moreover, the proof corresponding to each document is logarithmic in size with respect to the total number of documents occurring int the tree.

The properties of Merkle trees imply that committing to a Merkle root is a commit to all documents occurring in the tree, which can be arbitrarily many. Given a set $d_1,\ldots ,d_k$ of documents, a Merkle tree corresponding to these documents, also denoted $MT(d_1,\ldots ,d_k)$, is constructed by grouping the elements in pairs (the final element being duplicated in case that $k$ is odd), computing the hash value of the concatenation of each pair, and applying this recursively to create a complete binary tree whose leaves are labelled with these hash values. Each node whose childern are labelled with $x_1$ and $x_2$ is then labelled with the value $h(x_1 || x_2)$, with $||$ denoting concatenation. We illustrate this construction in Figure~\ref{merkle_fig}. To show that one particular document is in the Merkle tree, we just need to provide the list of siblings of nodes that appear on the path from the root of the tree to this document. For instance, assuming knowledge of the hash value at the root we can prove that the document $d_1$ belongs to the tree from Figure~\ref{merkle_fig} just by providing $d_1$, $x_2$ and $h(x_3||x_4)$. This way we can compute $x_1=h(d_1)$, the value $h(x_1||x_2)$, and the value of the root node.

A final interesting property of Merkle trees is that if we sort the hashes of documents, we can furthermore provide a logarithmic proof to show that a document does not belong to the tree.

\begin{figure}
\resizebox{\columnwidth}{!}{
\centering
\begin {tikzpicture}[-latex ,auto ,node distance =1 cm and 1.8cm ,on grid, semithick, state/.style ={ rectangle , draw, minimum width =0.9 cm}]
% First floor 
\node[state] (d1) {$d_1$};
\node[state] (d2) [right =3cm of d1] {$d_2$};
\node[state] (d3) [right =3cm of d2] {$d_3$};
\node[state] (d4) [right =3cm of d3] {$d_4$};
% Second floor
\node[state] (h1) [above =1.25 of d1] {$x_1=h(d_1)$};
\node[state] (h2) [above =1.25 of d2] {$x_2=h(d_2)$};
\node[state] (h3) [above =1.25 of d3] {$x_3=h(d_3)$};
\node[state] (h4) [above =1.25 of d4] {$x_4=h(d_4)$};
% Third floor
\node[state] (h12) [above right =1.25 and 1.5 of h1] {$h(x_1 || x_2)$};
\node[state] (h34) [above right =1.25 and 1.5  of h3] {$h(x_3 || x_4)$};
% Root
\node[state] (root) [above right =1.25 and 3 of h12] {\small Root};

% First floor paths
\path (h1) edge node[left]{}(d1);
\path (h2) edge node[left]{}(d2);
\path (h3) edge node[left]{}(d3);
\path (h4) edge node[left]{}(d4);

% Second floor paths
\path (h12) edge node{} (h1);
\path (h12) edge node{} (h2);
\path (h34) edge node{} (h3);
\path (h34) edge node{} (h4);

% Third floor paths
\path (root) edge node{} (h12);
\path (root) edge node{} (h34);

\end{tikzpicture}
}
\caption{A Merkle tree with document list $[d_1,d_2,d_3,d_4]$ and hash function $h$. Here, the root contains the value $h(h(x_1||x_2)||h(x_3||x_4))$, which is cryptographically linked to all documents.}% In order to prove that $d_4$ is present on the tree, one needs to check the validity of the blockchain $[x_4,h(x_3||x_4),\mathrm{root}]$ with knowledge of $x_3$ and $h(x_1||x_2)$ only.}
\label{merkle_fig}
\end{figure}

\medskip
\noindent{\bf Basic steps to gain trust.} Now that we understand the basic data structures that guarantee immutability, we can show how they are used to provide compact certificates showing that no data has been manipulated. The basic idea is quite simple: the government will keep on publishing their data as they do at the present, but at regular intervals, they will also publish a compact cryptographic digest of the new data. Assume that $d_1,d_2,\ldots ,d_k$ is the list of the documents published during one working day and that they are published in order. At the end of the day, the government will create a Merkle tree $MT(d_1,\ldots ,d_k)$ as previously described. At this point, the government publishes $MT(d_1,\ldots ,d_k)$, together with the cryptographic signature of this data corresponding to the public key under which the government agreed to publish their data. The fact that the signature matches the data in the Merkle tree and the government's public key, allows anyone to easily verify that the published root is legitimate. Moreover, if someone is only interested in certain documents, he can dismiss all other documents and simply keep the cryptographic proof that links the root of the Merkle tree with his document of interest.

As explained above, anyone who witnesses the above-described process will have an irrefutable proof of the exact set of documents published that day (assuming that the Merkle root cannot be tampered with). If the Merkle root published by the government does not match the one computed using this method, a verifier can know for certain that the government is not accting correctly. Moreover, if the government published the correct Merkle root but later on states any false fact about the data published that day, a verifier will be able to provide an irrefutable proof, exposing the government.

Notice that if we want to be able to detect which particular documents were modified, we would have to store all documents' hashes in the same order as they were used for constructing the tree. The size of this will be tens of megabytes for a list of one million documents when using SHA-256 as the hash function, so one might argue that even this is a reasonable amount of storage needed to pinpoint which particular documents had been manipulated. Moreover, even if the verifier does not have any information but the correct Merkle root, the government should be able to provide a succinct proof for the correctness of every document that was indeed part of the Merkle tree.
%One can further restrict the amount of needed storage by storing fewer layers of the Merkle tree (\ie if one trims the original leaves the size will roughly half), but also losing the granularity at which manipulations can be detected. 
%For instance, if we store the tree up to the next to last level, we can check if one of the two adjacent documents in the list $d_1,\ldots ,d_n$ have been modified, but will not be able to say which one (Figure \ref{fig-trim}). 
%For example, if we only store the root and its two children of the Merkle tree from Figure \ref{merkle_fig}, we can detect if there was a modification in the pair $d_1,d_2$, or the pair $d_3,d_4$, but we will not be able to say which of the two documents in the pair was modified.
%Based on the volume of the published data, the government might also wish to publish these partial trees and sign them, in order to make it easier for a person to store this information.
%Martin: esta discusión no me parece muy clara. Si yo guardo el hash de una concatenación no puedo verificar si alguien modificó algo, porque no tengo idea qué había debajo de mi hash.

It is possible to do better by utilizing the same idea that Bitcoin uses to store its blocks \cite{whitepaper}, and encode and/or publish the hash of the previous day's data in the current day's digest. That is, we can create a blockchain of the Merkle roots published each day. This would mean that storing one day's Merkle root (or the entire tree) would not only ensure that the data of that particular day can not be manipulated, but it would provide the same good properties for all data in the past. Since each tree would also contain the hash value of the previous day's digest, one can simply follow this chain day after day, and see if any of the hash values do not match. 
This simple concept is illustrated in Figure \ref{fig-merkle-chain}. Notice that there are two structures that are connected here. On the bottom level, we have the Merkle tree constructed from the day's documents. The top level blockchain is then constructed using the day's Merkle root as its data. That is, each element of this blockchain contains the Merkle root of the current day, plus the hash value of the previous day's element. The only caveat of this structure is that now certificates might become linear in the number of blocks in the blockchain. Although this can be avoided (e.g. by using skip lists) we do not describe the details here.

In this approach the government would publish, apart from the documents themselves, the Merkle root and the last block of the blockchain. In turn, each block will contain the day's Merkle root as its data, plus the hash of the (entire) previous block.

\begin{figure}
\centering
\begin {tikzpicture}[-latex ,auto ,node distance =1 cm and 2.5 cm ,on grid, semithick, state/.style ={ rectangle , draw, minimum width =0.9 cm}, substate/.style ={ rectangle , draw, minimum width =0.2 cm}]

\usetikzlibrary{shapes}
\node (x1) {};
\node (x2) [right = 2.7cm of x1]{};
\node (x3) [right = 2.7cm of x2]{};
\node (etc) [right = 1.8cm of x3]{};

\node[state] (r1) [below =of x1]{\scriptsize$\mathit{MR}_1$};
\draw [draw=black!50] ($(r1.south west) - (0.4,0.1cm)$) rectangle +(1.75cm, +1.2cm);
\node (h1) [above =15pt of r1]{$\bot$};
\node (bl1) [above =30pt of r1]{$B_1$};

\node[state] (r2) [right = 2.7 cm of r1]{\scriptsize$\mathit{MR}_2$};
\draw [draw=black!50] ($(r2.south west) - (0.4,0.1cm)$) rectangle +(1.75cm, +1.2cm);
\node (h2) [above =15pt of r2]{$h(B_1)$};
\node (bl2) [above =30pt of r2]{$B_2$};


\node[state] (r3) [right = 2.7cm of r2]{\scriptsize$\mathit{MR}_3$};
\draw [draw=black!50] ($(r3.south west) - (0.4,0.1cm)$) rectangle +(1.75cm, +1.2cm);
\node (h3) [above =15pt of r3]{$h(B_2)$};
\node (bl3) [above =30pt of r3]{$B_3$};

% First childs
\node[substate] (c11) [below left  = 0.9 of r1]{};
\node[substate] (c12) [below right = 0.9 of r1]{};
\node[substate] (c21) [below left  = 0.9 of r2]{};
\node[substate] (c22) [below right = 0.9 of r2]{};
\node[substate] (c31) [below left  = 0.9 of r3]{};
\node[substate] (c32) [below right = 0.9 of r3]{};

% Second childs
\node[substate] (c111) [below left  = 0.6 and 0.2 of c11]{};
\node[substate] (c112) [below right = 0.6 and 0.2 of c11]{};
\node[substate] (c121) [below left  = 0.6 and 0.2 of c12]{};
\node[substate] (c122) [below right = 0.6 and 0.2 of c12]{};

\node[substate] (c211) [below left  = 0.6 and 0.2 of c21]{};
\node[substate] (c212) [below right = 0.6 and 0.2 of c21]{};
\node[substate] (c221) [below left  = 0.6 and 0.2 of c22]{};
\node[substate] (c222) [below right = 0.6 and 0.2 of c22]{};

\node[substate] (c311) [below left  = 0.6 and 0.2 of c31]{};
\node[substate] (c312) [below right = 0.6 and 0.2 of c31]{};
\node[substate] (c321) [below left  = 0.6 and 0.2 of c32]{};
\node[substate] (c322) [below right = 0.6 and 0.2 of c32]{};

\node (etc111) [below = 0.3 of c111]{$\vdots$};
\node (etc112) [below = 0.3 of c112]{$\vdots$};

\node (etc121) [below = 0.3 of c121]{$\vdots$};
\node (etc122) [below = 0.3 of c122]{$\vdots$};

\node (etc21) [below = 0.9 of c21]{$\vdots$};
\node (etc22) [below = 0.9 of c22]{$\vdots$};
\node (etc31) [below = 0.9 of c31]{$\vdots$};
\node (etc32) [below = 0.9 of c32]{$\vdots$};

% Documents

\node (ghost1) [below left = 1.2 and 0.3 of c11]{};
\node (ghost2) [right = 1 of ghost1]{};
\node (ghost3) [right = 1 of ghost2]{};

\node (d1) [below left  = 0.8 and 0.3 of ghost1]{$d_1$};
\node (d2) [right = 0.5 of d1]{$d_2$};
\node (d3) [below left  = 0.8 and 0.25 of ghost2]{$d_3$};
\node (d4) [right = 0.5 of d3]{$d_4$};
\node (d5) [right = 0.5 of d4]{...};
\node (d6) [right = 0.5 of d5]{$d_k$};

%\path (r1) edge node {} (x1);
%\path (r2) edge node {} (x2);
%\path (r3) edge node {} (x3);
%\path (r3) edge node {} (x3);

%\path (x1) edge node {} (x2);
%\path (x2) edge node {} (x3);
%\path (x3) edge node {} (etc);

\path (r1) edge node {} (c11);
\path (r1) edge node {} (c12);
\path (r2) edge node {} (c21);
\path (r2) edge node {} (c22);
\path (r3) edge node {} (c31);
\path (r3) edge node {} (c32);

\path (c11) edge node {} (c111);
\path (c11) edge node {} (c112);
\path (c12) edge node {} (c121);
\path (c12) edge node {} (c122);

\path (c21) edge node {} (c211);
\path (c21) edge node {} (c212);
\path (c22) edge node {} (c221);
\path (c22) edge node {} (c222);

\path (c31) edge node {} (c311);
\path (c31) edge node {} (c312);
\path (c32) edge node {} (c321);
\path (c32) edge node {} (c322);

\path (ghost1) edge node {} (d1);
\path (ghost1) edge node {} (d2);
\path (ghost2) edge node {} (d3);
\path (ghost2) edge node {} (d4);
\path (ghost3) edge node {} (d6);

\node (ar1) [above right = 10pt and 0.77cm of r1] {};
\node (ar2) [above right = 10pt and 1.95cm of r1] {};
\node (ar3) [above right = 10pt and 3.47cm of r1] {};
\node (ar4) [above right = 10pt and 4.65cm of r1] {};

\path (ar2) edge node {} (ar1);
\path (ar4) edge node {} (ar3);

\end{tikzpicture}
\caption{Each day, a Merkle tree is constructed using that day's published documents $d_1,d_2,\dots, d_k$. The root of this tree is then hashed along with the hash value of the previous day, and the output value is broadcast along with the published documents. The result is a monitorable blockchain $[B_1,B_2,\dots]$.}
\label{fig-merkle-chain}
\end{figure}



\medskip
\noindent{\bf Increasing credibility.} Of course, the solution we just proposed suffers from a similar issue as just publishing government data without the additional blockchain and the Merkle trees. Namely, if no one monitors and keeps track of the blockchain or Merkle trees, the government is free to change the data, recompute the new Merkle trees and the blockchain, and publish them as if they were the original ones. Similarly, if someone does possess a single document they wish to show was changed, even if they have the Merkle tree for that day, it would still be a dispute between the government (claiming that the data that is presently published is the correct one) and the document holder, so it could be difficult to decide whose claim is the actual truth.

A rather simple solution to this problem is to involve into this process non-profit organizations that wish to monitor how transparent is the government in their data publishing practices. More precisely, such organizations would guard some degree of the meta information published by the government, and periodically download the government's documents for some particular day, compute their hash values, and check that the Merkle tree obtained in this way matches the one published by the government. In particular, once the government publicly announces its schema for publishing the data and additional security certificates (Merkle trees and blockchains) it would be simple to build an automated tool that helps organizations verify the correctness of the data-publishing process. Depending on  the organization's capacity, the tool could provide various levels of storage and verification requirements. In particular, the tool would support:
\begin{itemize}
\item Storing only the hash of the last block from the solution proposed in Figure \ref{fig-merkle-chain}. This is a very lightweight solution that can be supported even on mobile devices, as it requires storing only one hash value per day. Using this information, a day's data can be verified for modifications at a later date by downloading and hashing it. We can also detect whether the data on any day previous to this one has been modified.
\item Storing the entire Merkle tree, plus the top level blockchain of Merkle roots. As explained above, apart from allowing to detect whether the data of some particular day has been modified, this also allows us to  tracing which documents suffered changes.
%\item A hybrid solution storing the Merkle tree up to some level, allowing less granularity than storing the entire tree.
%Martin: Tengo la misma preocupación q arriba.
\item In terms of verification, the tool would provide support to verify the data of a particular day by downloading it, hashing it, and check that the resulting Merkle tree is correct. Additionally, the tool could verify backwards compatibility by also following the hashes of previous day's data, and checking that they match.
\end{itemize}

Note that in terms of storage, none of these solutions would be very expensive, and can in a sense serve as guarantors of the published data on government's pages. Additionally, if the amount of these organizations is sufficiently large, and in particular if they are not collaborating, it becomes unfeasible for a malignant agent to subvert all of their data. Furthermore, the fact that the lightest level of this solution could even be stored on smartphones makes it possible for interested individuals to guard the security certificates and check their veracity later, or effectively detect malicious activities.
